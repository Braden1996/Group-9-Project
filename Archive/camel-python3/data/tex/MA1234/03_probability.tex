% !TEX root = main.tex
%======================================================================
\chapter{Probability}\label{chap:prob}
%======================================================================

%----------------------------------------------------------------------
\section{Probability measures}
%----------------------------------------------------------------------
Probability is defined to be a \emph{function} that assigns numerical value to random events.

\begin{definition}
Let $\Omega$ be the sample space of some random experiment, and let $\mathcal{F}$ be a field of sets over $\Omega$. A \emph{probability measure} on $(\Omega,\mathcal{F})$ is a function 
\[
\begin{array}{rccl}
	\prob:	& \mathcal{F}	& \to	& [0,1] \\[1ex]
			& A				& \mapsto	& \prob(A)
\end{array}
\]
such that $\prob(\Omega) = 1$, and for any countable collection of pairwise disjoint events $\{A_1,A_2,\ldots\}$,
\[
\prob\left(\bigcup_{i=1}^\infty A_i\right) = \sum_{i=1}^{\infty} \prob(A_i).
\]
The triple $(\Omega,\mathcal{F},\prob)$ is called a \emph{probability space}.
\end{definition}

\begin{remark}
\bit
\it The second property is called \emph{countable additivity}.
\it The number $\prob(A)$ is called the \emph{probability} of event $A\in\mathcal{F}$.
\eit
\end{remark}

% example: die
\begin{example}
Consider a random experiment in which a fair six-sided die is rolled once.
\bit
\it A suitable sample space for the experiment is $\Omega=\{1,2,3,4,5,6\}$.
\it A suitable field of events for the experiment is the power set, $\mathcal{F} = \mathcal{P}(\Omega)$.
\it Because the die is fair, a suitable probability measure is given by the function
\[\begin{array}{rcl}
\prob:	\mathcal{F} & \to 		& [0,1] \\
		A			& \mapsto 	& \frac{1}{6}|A|, \qquad\text{where $|A|$ denotes the cardinality of $A$.}

\end{array}\] 
\eit

\begin{tabular}{lcc}
\underline{Event}				& \underline{Subset} & \underline{Probability} \\
The outcome is the number $1$.	& $\{1\}$ & $\prob(A) = 1/6$\\
The outcome is an even number.	& $\{2,4,6\}$  & $\prob(A) = 3/6$\\
The outcome is even but does not exceed $3$.	& $A = \{2,4,6\}\cap\{1,2,3\}$  & $\prob(A) = 1/6$\\
The outcome is not even			& $A = \Omega\setminus\{2,4,6\}$ & $\prob(A) = 3/6$
\end{tabular}
\end{example}

% example
\begin{example}
A fair six-sided die is rolled once. If we are only interested in whether the outcome is an odd or even number, we can take
\bit
\it Sample space: $\Omega=\{1,2,3,4,5,6\}$,
\it Events: $\mathcal{F} = \big\{\emptyset,\{1,3,5\},\{2,4,6\},\{1,2,3,4,5,6\}\big\}$
\it Probability measure: $\prob(\emptyset)=0$, $\prob(\{1,3,5\})=1/2$, $\prob(\{2,4,6\})=1/2$, $\prob(\{1,2,3,4,5,6\})=1$.
\eit
\end{example}

%----------------------------------------------------------------------
\section{Properties of probability measures}
%----------------------------------------------------------------------

% theorem: properties of probability measures
\begin{theorem}[Properties of probability measures]\label{thm:properties_of_probability_measures}
Let $(\Omega,\mathcal{F},\prob)$ be a probability space, and let $A,B\in\mathcal{F}$. 
\ben
\it Complementarity: $\prob(A^c) = 1 - \prob(A)$.
\it $\prob(\emptyset) = 0$,
\it Monotonicity: if $A\subseteq B$ then $\prob(A)\leq \prob(B)$.
\it Addition rule: $\prob(A\cup B) = \prob(A) + \prob(B) - \prob(A\cap B)$.
\een
\end{theorem}

% proof
\begin{proof}
\ben
\it % complementarity
Since $A\cup A^c=\Omega$ is a disjoint union and $\prob(\Omega)=1$, it follows by additivity that 
\[
1 = \prob(\Omega) = \prob(A\cup A^c) = \prob(A) + \prob(A^c).
\]
\it % emptyset
Since $\emptyset=\Omega^c$ and $\prob(\Omega)=1$, it follows by complemenarity that
\[
\prob(\emptyset) = \prob(\Omega^c) = 1 - \prob(\Omega) = 1 - 1 = 0.
\]
\it % monotonicity
Let $A\subseteq B$ and let us write $B = A\cup (B\setminus A)$. 

Since $A$ and $B\setminus A$ are disjoint sets, it follows by additivity that
\[
\prob(B) = \prob\big[A\cup (B\setminus A)\big] = \prob(A) + \prob(B\setminus A).
\]
Hence, because $\prob(B\setminus A)\geq 0$, it follows that $\prob(B) \geq \prob(A)$.

\it % addition rule
Let us write:
\bit
\it $A\cup B = (A\setminus B) \cup (B\setminus A) \cup (A\cap B)$
\it $A 		 = (A\setminus B) + (A\cap B)$
\it $B 		 = (B\setminus A) + (A\cap B)$
\eit
These are disjoint unions, so by additivity, 
\bit
\it $\prob(A\cup B) = \prob(A\setminus B) + \prob(B\setminus A) + \prob(A\cap B)$
\it $\prob(A) 		= \prob(A\setminus B) + \prob(A\cap B)$
\it $\prob(B)		= \prob(B\setminus A) + \prob(A\cap B)$
\eit
Hence $\prob(A\cup B) = \prob(A) + \prob(B) - \prob(A\cap B)$, as required.
\een
\end{proof}


%----------------------------------------------------------------------
\section{Exercises}
%----------------------------------------------------------------------

\begin{exercise}
\begin{questions}
\question % bookwork
What does it mean to say that $\prob$ is a probability measure over $(\Omega,\mathcal{F})$?
\begin{answer}
Bookwork. The symbols $\Omega$ (sample space) and $\mathcal{F}$ (field of events) should be defined before giving the definition of $\prob$.
\end{answer}
\question % subadditivity
Show that $\prob(A\cup B) \leq \prob(A)+\prob(B)$ for any two events $A$ and $B$.
\begin{answer}
First we express $A$, $B$ and $A\cup B$ as disjoint unions: 
\begin{align*}
A		& = (A\cap B^c)\cup (A\cap B) \\ 
B		& = (B\cap A^c)\cup (A\cap B) \\ 
A\cup B	& = (A\cap B^c)\cup (A\cap B) \cup (B\cap A^c)
\end{align*}
By the additivity property of probability measures,
\begin{align*}
\prob(A) 		& = \prob(A\cap B^c) + \prob(A\cap B) \\
\prob(B) 		& = \prob(B\cap A^c) + \prob(A\cap B) \\
\prob(A\cup B) 	& = \prob(A\cap B^c) + \prob(A\cap B) + \prob(B\cap A^c) \\
\end{align*}
From here, it follows that $\prob(A\cup B)=\prob(A)+\prob(B)-\prob(A\cap B)$, and because $\prob(A\cap B)\geq 0$ for any two events $A,B\mathcal{F}$, we see that $\prob(A\cup B) \leq \prob(A)+\prob(B)$, as required.
\end{answer}
\question % numerical
Let $A$ and $B$ be events such that $\prob(A)=0.4$, $\prob(B)=0.5$ and $\prob(A~\cup~B)=0.8$.\par
Compute the following probabilities:
\begin{parts}
\part 
$\prob(A\cap B)$.
\begin{answer}
$\prob(A\cap B) = \prob(A) + \prob(B) - \prob(A\cup B) = 0.4 + 0.5 - 0.8 = 0.1$.
\end{answer}
\part 
$\prob(A\cup B^c)$.
\begin{answer}
$\prob(A\cup B^c) = 1 - \prob(B\setminus A) = 1 - \big[\prob(B)-\prob(A\cap B)\big] = 1 - 0.4 = 0.6$.
\end{answer}
\end{parts}
\end{questions}
\end{exercise}


%----------------------------------------------------------------------
\section{Assignments}
%----------------------------------------------------------------------
%---------------------------------------
\begin{singlechoice}\label{sc:beatles}
\begin{questions} 
\question Who is the odd one out?  
	\begin{choices} 
	\choice John %\begin{feedback} Wrong \end{feedback}
	\choice Paul
	\choice George
	\correctchoice Bingo
	\end{choices} 
\end{questions}
\end{singlechoice}
%----------------------------------------

%---------------------------------------
\begin{multiplechoice}\label{mc:basic_arithmetic}
\begin{questions} 
\question Which of the following statements are correct?  
	\begin{checkboxes} 
	\correctchoice $1 + 1 = 2$
	\choice $1 + 1 = 3$
	\choice $2 + 2 = 3$
	\correctchoice $2 +2 = 4$
	\end{checkboxes} 
\end{questions}
\end{multiplechoice}
%----------------------------------------

\begin{homework}\label{hw:probability}
\begin{questions}
\question
Let $A$ and $B$ be random events, with probabilities $\prob(A) = 1/2$ and $\prob(B) = 3/4$. 
\begin{parts}
\part 
Show that $\displaystyle\frac{1}{4}\leq \prob(A\cap B)\leq\frac{1}{2}$.
\begin{answer}
$A\cap B\subseteq A$ and $A\cap B\subseteq B$ means that:
\[
\prob(A\cap B) \leq \min\big\{\prob(A),\prob(B)\big\} = \displaystyle\frac{1}{2}.
\]
Furthermore, $\prob(A\cup B)\leq 1$ means that:
\[
\prob(A\cap B) = \prob(A)+\prob(B)-\prob(A\cup B) \geq \displaystyle\frac{1}{4}.
\]
\end{answer}
\part 
Show that $\displaystyle\frac{3}{4}\leq \prob(A\cup B)\leq 1$.
\begin{answer}
$A\subseteq A\cup B$ and $B\subseteq A\cup B$ means that:
\[
\prob(A\cup B) \geq \max\big[\prob(A),\prob(B)\big] = \displaystyle\frac{3}{4}.
\]
Furthermore, $\prob(A\cup B)\leq 1$ means that:
\[
\prob(A\cup B) \leq \min\{1,\prob(A)+\prob(B)\} = 1.
\]
\end{answer}
\end{parts}
\end{questions}
\end{homework}

%======================================================================
\endinput
%======================================================================
