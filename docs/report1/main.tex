%% Website of the LaTeX Gods: https://en.wikibooks.org/wiki/LaTeX
\documentclass[a4paper, 12pt, titlepage]{article}
%%%%%%%%%%%%%%%%%%%%%%%%%%%%%%
% STARTING DOCUMENT PREAMBLE %
%%%%%%%%%%%%%%%%%%%%%%%%%%%%%%

\usepackage[utf8]{inputenc}
\usepackage{anyfontsize}
\usepackage[usenames, dvipsnames]{color}
\usepackage{titlesec}
\usepackage{url}
\usepackage{enumitem}

\title{First Report}
\author{Group 9}

\newcommand{\makeColourful}{\color{MidnightBlue}}
\newcommand{\makeGrey}{\color{Gray}}
\newcommand{\makeBlack}{\color{Black}}
\renewcommand*\contentsname{\fontsize{25}{30}\selectfont Table of Contents}
\renewcommand{\thesection}{\Roman{section}}

\titleformat{\section}{\LARGE \bfseries \makeColourful}{\thesection}{0.7cm}{}[{\hrule}]
\titleformat{\subsection}{\Large \bfseries \raggedleft \makeColourful}{}{0}{}[]
\titleformat{\subsubsection}{\large \bfseries \raggedleft \makeColourful}{}{0}{}[]
\titleformat{\subsubsubsection}{\normalsize \bfseries \raggedleft \makeColourful}{}{0}{}[]
\titleformat{\subsubsubsubsection}{\normalsize \bfseries \raggedleft \makeColourful}{}{0}{}[]

%%%%%%%%%%%%%%%%%%%%%%%%%%%%
% ENDING DOCUMENT PREAMBLE %
%%%%%%%%%%%%%%%%%%%%%%%%%%%%
\begin{document}

\begin{titlepage}
\newcommand{\HRule}{\rule{\linewidth}{1.5mm}}
\center

\begin{minipage}{0.4\textwidth}
\begin{flushleft}
\textbf{\large CM2305}
\\
\emph{\large Group Project}
\end{flushleft}
\end{minipage}
~
\begin{minipage}{0.4\textwidth}
\begin{flushright}
\textbf{\large Module Leader}
\\
\emph{\large Dr Frank Langbein} 
\end{flushright}
\end{minipage}
\\[2cm]

{\makeColourful \HRule} 
\\[2cm]
{\fontsize{90pt}{0pt}\selectfont \makeColourful{\textsc{First Report}}}
\\[1.5cm]
A \LaTeX\ document
{\makeColourful \HRule}
\\[6cm]

\textsc{\Large Group 9}
\\[0.5cm] 
\textsc{Aimi Daros, Braden Marshall, Cameron Fish, Lucy Robertshaw, Mariza Celliers, Ryan Flynn, Sh’kyra Jordon, Skye Watkins, Thomas Rice.}

\end{titlepage}

\newpage
\tableofcontents
\newpage

\subsection{Revised Requirements}
	For the development of CAMEL we have decided to revise some of the existing requirements that we proposed in the first report. We have decided to change a few of the requirements as we have decided to go with a different approach to development than we originally intended in the first report.\\
	
	\textbf{New Requirements}
	\begin{itemize}
		\item New parser implementation
		\item Re-factoring of existing codebase
	\end{itemize}
	
\documentclass[12pt]{article}

\usepackage{geometry}
\geometry{a4paper}

\usepackage{url}

\linespread{1.2}

\setlength{\parindent}{0pt}
\setlength{\parskip}{1em}

\begin{document}
	\begin{titlepage}
		\newcommand{\HRule}{\rule{\linewidth}{0.5mm}}

		\center

		\textsc{\LARGE Cardiff University}\\[1.5cm]
		\textsc{\Large Computer Science}\\[0.5cm]
		\textsc{\large CM2305: System's Design \& Group Project}\\[0.5cm]

		\HRule \\[0.4cm]
		\textsc{\Large \textbf{CAMEL}}\\[0.1cm]
		\textsc{\Large \textbf{CA}rdiff \textbf{M}athematics \textbf{E-L}earning}\\[0.7cm]
		{\huge\bfseries Requirements}\\[0.4cm]
		\HRule \\[1.5cm]

		\begin{minipage}{0.4\textwidth}
			\begin{flushleft} \large
				\emph{Authors:}\\
				\mbox{Mariza \textsc{Celliers}}, \mbox{Ryan \textsc{Flynn}}, \mbox{Braden \textsc{Marshall}}
			\end{flushleft}
		\end{minipage}
		~
		\begin{minipage}{0.4\textwidth}
			\begin{flushright} \large
				\emph{Clients:} \\
				\mbox{Stuart \textsc{Allen}}, \mbox{Dafydd \textsc{Evans}}
			\end{flushright}
		\end{minipage}\\[3cm]

		{\large \today}\\[2cm]

		\vfill
	\end{titlepage}


	\tableofcontents

	\newpage
	\section{System Scope}
	The proposed project that has been put forward to us, is to update and complete the implementation of the current CAMEL system. The existing codebase needs additional functionality, which includes lexical comparisons, logging of student answers and deadline functionality . On top of providing a improved system, documentation and unit tests will also be included.

	To address the clients proposal, CAMEL 2.0 will issues a code standard. This will be PEP8 Python standard, and will be the first priority to implement to the system. Additionally, exception handling will also be added as well as extra code to fix any current bugs that reside. Besides maintaining the code, documentation for a developer and administrator will be added. All of this will be part of an ongoing process during development, with each being equally important.   
	
	Addressed functionality (see  functional requirements) will be implemented in agile sprints, the length of the sprint cycle for functionality is dependant on the complexity. For example, lexical comparisons may take longer than adding exception handling, due to the complexity that is required to add this into the codebase. Other functionality, such as database re-design and break down of core functionality will be allotted a time frame during the sprint cycle meetings. The plans for the cycles will be drawn up every week during the implementation stage.   
	
	Tests will be added prior to writing the code, though will be written in Django's test framework first. Additional or 'conceptual' functionality will be addressed last, to ensure high code quality throughout. However, due to a fixed time frame, we will only be implementing functionality that is mentioned in the requirements. No additional requirements later on will be added.          	
	\section{Assumptions}
	Due to the fixed nature of the current system, there are some aspects that we will need to assume. Firstly, we are assuming that the functionality we are developing for is single threaded. This assumption is being made as the time scale to add threaded functionality is non-existent. Furthermore, we are developing the system on Django's development server, which cannot simulate a multi-threaded environment. 
	
	A further assumption we are making is that there will be no additional requirements added when implementation occurs, and these requirements are final. Given the time scale we have to make the system, it would be unlikely that any additional requirements could be completed to a high standard. Any additional functionality will be dropped due to development scope. 
	
	Finally, we area assuming that the use of third party software to handle aspects that are out of scope can be integrated into the system. Third party applications that may be include:
	\begin{itemize}
		\item JQuery - JavaScript Library
		\item Bootstrap - CSS Library
		\item Additional Python modules
	\end{itemize}
	\section{Functional Requirements}
	\subsection{Add methods of checking and comparing student answers from questions. - R}
	To meet the initial requirement of lexical analysis, functionality will need to be added to current student answers to be compared. Functionality that can be included may be graphs and statistics. With this functionality, lecturers are able to see possible trends with particular pieces of work, and see if there are any issues. This functionality can also be expanded as a plagiarism checker, by looking for common answers. The depth of functionality that could be provided is limited to the development scope.
	\subsection{Logging of Student Answers}
	As the primary goal of the system is to allow members of staff to upload tests - in the form of LaTeX documents - with 'interactive' answering features, for the students to access, answer and submit; it seems only natural that these submissions are to be logged and then viewable by the supervising members of staff. A time-stamp should be attached to each submitted answer to allow a deeper understanding of student upload patterns.
	\subsection{Add deadlines to student homework questions which will prevent late submissions. - M}
	In the brief it specifies that there should be capabilities to ‘complete and submit homework assignments’ and our client has also outlined that a way of adding deadlines to homework questions is something that should be implemented on the system.
By being able to have deadlines on homework submissions, it would ensure the solutions of the students who miss the deadlines are not marked and their score is not added to their overall grade. It would also allow the grades of these students to not be included in the analysis and comparison of the classes grades.
After a deadline has passed for a specific coursework, the student should be able to see and write answers for the questions, but not be able to submit it.  And the lecturer should be able to see which students have not submitted.
A lecturer should also have the option of choosing not to set a deadline.

	\subsection{Provide functionality to only see modules a student is enrolled for. - R}
	This requirement will be a proof of concept rather than a fully functional feature, due to not having the permissions to access enrolment data. The concept will instead simulate some test data to assess the possibility of this feature being added. If the concept is a success, it will allow the UI for the student to be simple, as it will only show their enrolled modules. This adds to ease of use, however this feature is not detrimental to the final system. The functionality of these feature is dependent with time available.
	\subsection{Input Error Checking}
	In the current version of the system, when the contents of a LaTeX document is passed into the LaTeX parser, there is virtually no validation of this input. The input is wrongly assumed to be already correctly formatted. Although our client informed us that adding validation functionality for the staff uploaded LaTeX test documents was not a necessity, he did request that any arising errors should be safely dealt with i.e. not cause a system crash or corrupt the database.
	\subsection{Provide exception handling. - M}
	In the brief is specifically states that the existing codebase has no exception handling, and that this needs to be added. By speaking to our client this is a requirement that is of a high priority and will have to be implemented.
The exception handling in the system would enable us to find and fix bugs easily and quickly during production, it would also enable the code to be maintainable in the future and it would also allow for smooth running for the system.

	\subsection{Provide Unit Tests. - R}
	All code that is developed will utilise the Django’s already existing test framework, along with Python’s testing mechanics. This will be a continuous process in the development sprints. Though all tests will be created first, before code is written. This will give a good indication of the time each feature requires to be implemented. By having the tests drawn up first, we allow the requirement to be flexible, and have the ability to make changes if issues occur. It also allows possible extensions to be added in the future, as test first development ensures high code quality. This will also meet the requirement proposed in the brief by including these tests. These tests can be utilised by future developers using Django.

	\newpage
	
	\section{Non-Functional Requirements}
	\subsection{Fix Bugs in the Current System}
	The current system, as it stands, is in no way acceptable for deployment. There exists an unknown amount of bugs and several poor design-choices which desperately require fixing for the project to be a success. If these issues are not dealt with accordingly, their impact will greatly hinder development and the system would be left unreliable and potentially dangerous.
	\subsection{Improve navigation reliability when moving between pages - B}
	When navigating around the current system, it seems that many of the hyper-links are incorrectly using relative address references - as oppose to absolute or referring to the root domain. This results in users of the system being miss-directed when navigating around the site. It even makes some areas of the site completely unacceptable without the user manually changing the file path.  This is likely due to the incorrect usage of Django's URL template tag\footnote{\url{https://docs.djangoproject.com/en/stable/ref/templates/builtins/#url}} and should be solvable without a huge amount of issues.
	\subsection{Ease of use - M}
	In the brief it states that ‘the way in which students and teachers interact with the system should also be investigated and the design modified accordingly’. Our client has also made it clear that they would like the system easy enough for the students and teachers to use so that they will not need documentation, and should be intuitive and straight forward to navigate and find the needed information easily.
	\subsection{Administrator and Developer documentation. - R}
	Documentation will be required to meet a proposed requirement. The main focus for the documentation will be for the developer, which will focus on the technical aspects. It will include details on functionality on particular files and certain standards in the code. The administrator documentation will be 'lightweight' and will include all the instructions and commands that are needed to operate the system. Though the developer documentation will be considered a higher priority over the administrator, due to the UI being self explanatory. The documentation will be accessed on the site itself for both developer and administrator. 
	\subsection{System Permissions}
	The system currently features two different types of users - students and staff members. It is required that various additional permission parameters are to be attached to the staff accounts, so that only they are able to access the sensitive areas of the system. These sensitive areas include:
	\begin{itemize}  
        \item LaTeX test document upload point
        \item LaTeX test document manage point
        \item List of the registered students
        \item List of the student's test results
    \end{itemize}
    Depending on time restraints, it is advisable that Django's automatic admin interface\footnote{\url{https://docs.djangoproject.com/en/stable/ref/contrib/admin/}} not be accessible by staff members. This is because it is an extremely powerful tool that could easily be misused and permanently damage the system's stored data.

	\subsection{Upgrade System for Python3. - M}
	After discussing with our client, upgrading to Python 3 is an optional requirement depending on how easy this change would be and how long that it would take us to change the existing codebase. 
We would want to move to python 3 as majority of the IO issues have been worked out, and python 2 will eventually become obsolete. By changing to python 3 it means that the system will be more maintainable as more people will know this release of python, and the system would be compatible with more devices for a longer period of time.

	\subsection{Have maintainable code. - R}
	Throughout the development cycle, the aim will be to update and add code that follows a set of standards. These standards will include comments, naming conventions and modularity of files. For this project, we have been instructed to use the PEP8 Python standard. This standard focuses on using underscores for naming conventions. Upon bringing these standards, it will ensure that future developers will be able to quickly understand the context of a code and continue to maintain high the system for the future.
	
	\subsection{Modify database to allow tables for Student, Teacher, Author, Admin. - M}
	Our client has also mentioned to add four more tables into the database to be able to be able to ensure no data is lost and kept in an organised manner for if it needs to be manually accessed. It would also ensure that these attributes can be easily referenced and compared throughout the system.
	
	\subsection{Break down the core code to become smaller 'applications' in Django. - M}
	After the third meeting with our client, he has specified specifically that he would like the code to be broken down and organised into the correct Django ‘applications’. By doing this the Code will be better organised and easier to understand by others therefore more maintainable. If the code was better split up into the applications, it would also cause code reuse to be easier throughout the application, later on if the system expands, or is released to be downloaded and used by the public.
\end{document}


\section{Acceptance Criteria}
Lorem ipsum dolor sit amet, consectetur adipiscing elit. Sed rutrum viverra nunc pellentesque finibus. Fusce id diam venenatis, tincidunt tellus sed, commodo sem.

\addcontentsline{toc}{subsection}{Acceptance Criteria for Functional Requirements}
\subsection*{Acceptance Criteria for Functional Requirements}
Lorem ipsum dolor sit amet, consectetur adipiscing elit. Sed rutrum viverra nunc pellentesque finibus. Fusce id diam venenatis, tincidunt tellus sed, commodo sem.

\addcontentsline{toc}{subsection}{Acceptance Criteria for nonfunctional Requirements}
\subsection*{Acceptance Criteria for nonfunctional Requirements}
Lorem ipsum dolor sit amet, consectetur adipiscing elit. Sed rutrum viverra nunc pellentesque finibus. Fusce id diam venenatis, tincidunt tellus sed, commodo sem.

\addcontentsline{toc}{subsection}{Benefit \& Risk Assessment}
\subsection*{Benefit \& Risk Assessment}
%Risk map outline to go before paragraph

In order to assess the risks of our project we included the use of a risk map. Firstly we identified the risk that could occur in the duration of our project and then described the effect it would have upon the group. We calculated the total risk by measuring the likelihood of the event occurring, against the impact it would have should it occur. The controls include the contingency, avoidance and how we would cope with the risk if it occurred. This helps us to maintain the project goals and control the overall risk. After the controls are implemented, we then measured the total risk again in the hope it would have improved. We followed the colour schemes of the risk map so it is more visually appealing and clearer to assess how the controls we implemented help. We aimed to achieve the colours of green and yellow for the total risk given the controls. We then analysed the benefits of having the risk controls in place for each risk, and explained the benefits as a whole.
%completed risk map to go after paragraph

\addcontentsline{toc}{subsection}{Relevant Quality Factors}
\subsection*{Relevant Quality Factors}

\subsubsection*{Flexibility and Extensibility} 

\subsubsection*{Maintainability and Readability} 

\subsubsection*{Performance and Efficiency} 

\subsubsection*{Scalability} 

\subsubsection*{Availability, Fault Tolerance and Reliability} 

\subsubsection*{Usability and Accessibility} 

\subsubsection*{Platform Compatibility and Portability} 

\subsubsection*{Testability and Manageability} 

\subsubsection*{Security} 

\subsubsection*{Functionality and Correctness} 

\section{LSEPI}
    When considering the creation of a system like Camel it’s easy to get lost amongst the development process and overlook assessment of the legal, social, ethical and professional issues (LSEPI) raised by the creation and use of such a system.
    \addcontentsline{toc}{subsection}{Intellectual Property}
    \subsection*{Intellectual Property}
    Camel is a system designed for use by academic institutions alone, liberating it from many legal constraints.
        
    With regards to this system, we need only to focus on 2 kinds of intellectual property; that of parties outside of the educational establishment and that of parties within. Though in Maths, works like formulas aren't copyright protected, sufficiently original algorithms, images, video clips (etc.) are.

    If, for example, a lecturer wishes to copy and, via the system, communicate example questions, exercises or diagrams from external sources to other members within his educational establishment (namely his pupils), he must make this copy solely for the purposes of education/instruction for a noncommercial purpose and accompany it (unless impossible due to reasons of impracticality or otherwise) with sufficient acknowledgement of the author. Since this copy will be available to others in a copyable format it is legally classified as an “accessible copy” and comes with the responsibility of ensuring it is used only for educational purposes. If it is not an extract of works by an educational establishment, it must also be accompanied by a statement that the copy was made under the Copyright Designs and Patent Act (1988).\cite[(Pt 1, Ch 3, §31B)]{CDaPA}

    It should also be noted that since this work is presented to members within and connected directly to an educational establishment by a teacher/pupil in the course of activities of educational establishment for the purpose of education/instruction for a noncommercial purpose, the presentation of this work is not considered public in the context of copyright infringement.

    This could prompt Camel software developers to integrate some sort of anti-screenshot/ anti-ctrl+x functionality to minimise the use of accessible copies of external works by pupils for non-educational purposes, though it should be considered that most material that maths lecturers would share with pupils could not be used in any other way but educational so perhaps rather than attempting to prevent duplication of this material, functionality could be incorporated to watermark screenshots with relevant acknowledgements or append it to strings copied with a browser’s “copy/paste” command.

    Intellectual property (in the form of their own words, images, audio clips etc.) of staff within the academic establishment contributed as module content/material could either be treated with the same respects as external parties as an individual author or as part of the establishment with non-individual rights, in which case works would be acknowledged as property of their academic establishment, for example, ”© Copyright protected Cardiff University”. 

    Another example of a party within an academic establishment using Camel that may contribute their intellectual property for use via the system is any pupil who submits their own work perhaps as part of an assessment, for example. In this case any use of their work for the sake of education, if complying with the conditions stipulated in the Copyright Designs and Patent Act (1988) as stated above, similarly cannot be deemed as copyright infringement.

    If intellectual property of one of these parties was used in any way that was not in the name of education (especially for commercial purposes, particularly that of which would generate profit that does not go to the owner) without proper acknowledgement affecting the party’s right to paternity of their work (i.e the right to be recognised as the owner) or the work was tampered with in some way that affected the work’s integrity it would and thus infringe copyright law\cite{LSEPI in Computer Tech} if not by the letter of the law then at least in spirit of the law. In this situation the party whose intellectual property rights have been infringed can seek to take legal action in the form of an injunction, interdict or other legal order to stop the infringing party from continuing to violate their rights by making further infringing use of their work. The party can also impel or even oblige the infringing party to remove or give up their copy of the infringing material and even claim damages from them.\cite{JiscLegal - Intellectual Prop.}
    
    Even if the spirit of the law does not create enough of a case for the owner to press charges of infringement, it should be noted that efforts should be given to protect the general right to paternity and integrity of an individual’s intellectual property as a gesture of goodwill, especially between and within establishments of education since we tread so dangerously, with regards to intellectual property, along the line between ethical and unethical practice, it is hard not to when dealing with so many new learners dealing with new technologies,\cite{JiscLegal - Intellectual Prop.} and since any member of an academic establishment is equally vulnerable to the bad faith of other members of an academic establishment, a two-way consideration benefits both parties in the long-run simply by maintaining a mutual respect between us all.
    \pagebreak
    \addcontentsline{toc}{subsection}{Data Protection}
    \subsection*{Data Protection}
    In regards to Camel’s database, legal, ethical and social issues could potentially arise around the security, processing and containment of personal and sensitive information. Users will trust Camel to take care of their personal and sensitive data, particularly that of their name, test results, unique school IDs/emails, etcetera which will be processed mainly via storing and statistically analysing. The possible repercussions (and solutions) to this process must be considered.

    Since we are dealing with trust, ethically, what information is being stored and how it is used must be taken into consideration; how essential its existence in the database is to the function of Camel must be weighed against the sensitivity of the data and the potential implications of its storage, particularly in the case of its leakage.

    Examples of sensitive information could range from anything between a party’s name or address to their learning difficulty or financial situation to their national insurance number or passport number. Given any piece of information about a person, it can be difficult to determine whether it could be classified as sensitive or not, one could ask the person in question or refer to common census but more commonly we err on the side of caution and treat all personal and sensitive information with the same level of caution, aside in the context of requesting legal consent and in the instance where personal information is deliberately shared (e.g. on a professional networking site known as LinkedIn, all members are expected to go by their real names).
    
    On the legal side of things, storage and processing of an individual’s data need only comply with the DPA (Data Protection Act 1998), this means that in the context of our Camel system, any personal information about an individual can only be stored for processing if it is necessary either for the performance of some contract the individual is part of, or if the individual gives consent to their information being used this way.\cite{DPA:tDPP:S1}\cite{DPA:tDPP:S2}
    
    \vspace{0.2cm}
    \hrule
    \begin{center}
        \small \em “\ldots personal information about an individual can only be stored for processing if it is necessary \ldots for the performance of some contract the individual is part of…”
    \end{center}
    \begin{flushright}
        \scriptsize E.g. a university may have a contract with a student to award him an academic degree of a class relating to his performance over the duration of a course and would thus have the legal right to store and process examination results throughout his course in order to fulfil this contract to a high standard.
    \end{flushright}
    \hrule
    \vspace{0.2cm}

    The DPA also stipulates that if an individual’s personal information is disclosed but consent to processing has not yet been received, it may still be processed if out of a legitimate interest by those it was disclosed to unless it is unwanted in caution of potential prejudice of the individual’s rights, freedoms or legitimate interests\cite{DPA:tDPP:S2} (i.e. a party suspects processing of this data will cause the individual in question to be unfairly prejudiced for reasons such as that of sex, learning difficulties, nationality, race, age etc.).

    \vspace{0.2cm}
    \hrule
    \begin{center}
        \small \em “\ldots[disclosed personal data that has not yet received processing consent] may still be processed if out of a legitimate interest by those it was disclosed to…”
    \end{center}

    \begin{flushright}
        \scriptsize E.g. Once receiving their grades from a module exam, a subject leader need not ask consent from his pupils to process their results to analyse the performance of the subordinate teacher who taught them the module content.
    \end{flushright}
    \hrule
    \vspace{0.2cm}

    In addition, any personal information considered sensitive can only be stored and processed lawfully if either explicit consent has been given for the information to be stored and processed this way or this data is already public by deliberate actions of the individual in question.\cite{DPA:tDPP:S3} This means that Camel can easily avoid legal disputes by adding into the terms and conditions of use for any user, pupil or otherwise, that any information gathered for the need of Camel’s function has the consent of the user to be attained, stored and processed. Though it should be noted that many users will skip over long terms and conditions to continue with their work, though this protects the establishment legally, on ethical grounds it is recommended that disclosure is repeated when that information is gathered.

    \vspace{0.2cm}
    \hrule
    \begin{flushright}
        \scriptsize For example, pupils could select an “opt-in” checkbox before a test to explicitly consent to timestamps, timing, results and solutions being stored, processed for analytical reasons and being used for educational purposes for future pupils either by being shown anonymously  to future classes or via anonymous peer marking.
    \end{flushright}
    \hrule
    \vspace{0.2cm}

    It is considered good practice (both in an ethical and legal sense) to fully disclose the user why and how their personal data will be processed, stored and used in an upfront, clear manner. 

    Any information/data stored/processed must be relevant and non-excessive relative to the reason it was stored for to be legally sound, it must also be kept accurate, up-to-date and not be kept for longer than necessary. There is also mention in the DPA of an obligation for measures to be put in place to protect this personal data from “unauthorised or unlawful processing … [and] accidental loss\ldots destruction\ldots, or damage\ldots”.\cite{DPA:tDPP:S1} This means that any personal data Camel stores in its database must either be manually updated regularly or done so automatically.

    \vspace{0.2cm}
    \hrule
    \begin{flushright}
        \scriptsize For example, Camel could include a regular updating and synchronisation process with an academic establishments Intranet system.
    \end{flushright}
    \hrule
    \vspace{0.2cm}

    It is also important that information that is not necessary for storage is not logged or saved after automatic deletion from the system’s cache.

    \hrule
    \begin{flushright}
        \scriptsize  For example, we may find need to store reference to which browser a pupil is using in case submitting glitches due to  browser incompatibility, this should be deleted after the pupil either submits their work successfully or is given a disclaimer on the screen reading something like “Having problems submitting? Try again on Chrome, Internet Explorer or Opera”.
    \end{flushright}
    \hrule
    \vspace{0.2cm}

    Considering the ethical, social and professional side of it, If personal data is stored within a database, it must be kept secure. We contribute to this by minimising the potential problems arising from a privacy breach with damage control. Any means by which a party with no essential need to view or use the data can view or use the data should be limited; only those with the required permissions should be able to see the data, and only those who are required to see the data should have those permissions. Anyone else should not be able to see this information under any circumstances. This emphasises the necessity of Camel’s need to be secure, air-tight unhackable security can be strived for by putting in place measures to remove the possibility of any user accidentally stumbling upon (or intentionally) this information, in fact these security measures should be able to resolve any issues that could result in a privacy breach, including glitches.

    Recording a users IP address would also come under jurisdiction of the data protection act. There may be a desire to store a pupil's IP addresses on login in order to help determine whether there has been a security breach. Recording an IP could be argued to be unnecessary and a breach of privacy since the risks associated with storing a user’s IP include its potential use to determine some sensitive information like, for example, their location during use of the system, which could indicate their home address or a place of leisure they frequent. This means storing a user’s IP within a database could be a potential violation of the spirit of the DPA given sufficient security is not implemented, or the information is not deleted as soon as it is not necessary.

    Since storing IPs and timestamps could be used in the name of security it could serve to be a more vital function for Camel, justifying it ethically and legally.

    \vspace{0.2cm}
    \hrule
    \begin{flushright}
        \scriptsize For example, catching multiple logins from separate locations/IPs into a user’s account could determine the existence of fraudulent practice by pupils or other users without the pupil’s permission, which would then allow the system to alert the user or an admin of a potential security issue and suggest a change of password.
    \end{flushright}
    \hrule
    \vspace{0.2cm}

    This functionally could be maintained in an ethical way if this data is stored only for a limited time before automatic deletion, perhaps until the pupil has received an official immutable mark for the work at the end of the year, module, semester or week, depending on the structure of the specific course.

    \addcontentsline{toc}{subsection}{Analysis}
    \subsection*{Analysis}
    The analysis of data pertaining to an individual can be controversial; many can perceive it as an objectionable invasive and insensitive systematic breach of privacy depending on how it is conducted and why, thus giving Camel’s analytical functions potential rise to social and ethical issues with regards specifically to data gathered about pupils.

    One functionality the client hints at desiring is the ascertainment of work groups by considering similarities in answers of pupils. On one hand, this could be perceived as socially unjust to pupils who feel they are being matched with others based on their weaknesses, but on the other hand if the same analysis was carried out for, and justified by, its contribution to detecting plagiarism it would raise next to no issues, but it should be noted that any evidence gathered in support of fraudulent submissions by pupils could be largely attributed to coincidence, especially in Mathematics. This weigh in against the advantages and disadvantages of such analysis prompts the inclusion of an opt-out option for pupils upon initial use of the system.

    Extending on this, a further social issue arises when a pupil is shown their results relative to the rest of their class. While some pupils may benefit from this by motivating them to excel further or by giving them gratification where it is due, some may be demotivated if their efforts have not yielded proportionate success potentially afflicting them detrimentally. This raises the question; is it ethical to show a pupil this being aware of the potential effect it could have? While it could be argued that its their own responsibility to manage their response to an accurate assessment of their progress, consideration could be given to those less able to manage that responsibility, particularly those who may not be underachieving relative to the rest of the class. This could be resolved by choosing not to show this to any pupil at all, in consideration of the possible benefits to a pupil’s success and in the spirit of full disclosure, it would probably be optimal to show these results only to those who wish to see it by for example requiring them to click a “show my results against the rest of the class” button first.

    Camel will time how long it takes students to complete assessments, this is necessary both in detecting plagiarism and in assessing the ability of the pupil, as to whether a student should be shown this information is somewhat debateable for reasons similar to that regarding their results relative to the rest of their class. However since most Intranet systems with online tests do this without any objection from staff or students, we will be implementing this regardless though it should be noted that in future we could allow this information to be viewed optionally very easily, or even remove the option all together since a pupil can effectively access the same information by timing themselves. These options can be explored should any issues arise in the future.

    \addcontentsline{toc}{subsection}{Marking}
    \subsection*{Marking}
    It has always been an issue of professionalism when unintentional bias by a teacher could affect the mark of a student, since the class of one's’ degree is so important in their migration to the professional world it is important Camel maintains the same cautionary measures other forms of course submission and marking do.

    Hardcopy assessment submissions are assessed as anonymous submissions by lecturers in order to meet a standard of care for a lecturer’s pupils. This should be mimicked in the functionality of Camel where any answers that must be manually marked by a member of staff should be done so without the identity of the pupil in question being revealed. It is best to do this with both formative and summative assessments but it should be acknowledged that this could potentially limit the ability of the lecturer to provide feedback tailored to the student since any helpful understanding of the pupil that the teacher may have is made redundant and also, such obtaining such understanding from the test results is inhibited by anonymity. This could be worked around if for example, anonymity was only maintained for summative assessment and not formative assessment which is, after all, primarily meant to provide practice for, aid the teacher in the education of, and give the teacher opportunity to give constructive feedback to their student. Alternatively the teacher could mark each question with the student’s identity hidden, only to be revealed automatically once the mark has been given or perhaps manually only in the instance the teacher feels it prudent to give individual feedback.

    Although it may seem abundantly clear, some students may think that being assessed through an online test means that their answers will be automatically marked and viewed by no one when in reality, answers that must be manually marked or those that a lecturer may choose to show as an example for future classes will be viewed by other people. In this case not only must all anonymity be maintained (unless perhaps in the latter case, a student wishes to be credited in which case perhaps an opt-in/opt-out button for anonymity can be included), but pupils must receive a disclaimer that for any open ended question, it may be viewed by a member of staff or by future students (though again, for the latter case, this is not a necessity and can be optional).

    An educationally beneficial means of marking (formative) assessment is peer marking, it teaches students to be critical of work thus giving them the ability to be critical of their own work should they try, and it lessens pressure and time constraints on a lecturer freeing his efforts up for meeting the educational needs of his students more effectively. This is something that students may object to, especially if their anonymity cannot be guaranteed, if they do not wish to be scrutinised by their peers. One way around this is to disclose to the student, before a test, that their results may be marked by their peers and perhaps even include an opt-out button so that their test will be marked by a teacher instead. It is also important that security measures be put in place to keep the students identity anonymous.

    \addcontentsline{toc}{subsection}{Conclusion}
    \subsection*{Conclusion}
    While this is a comprehensive discussion of the legal, social, ethical and professional issues regarding the proposed Camel system, but it should be noted this is not an exhaustive list and is mainly meant to serve the purpose of aiding further, more exhaustive discussion of Camel once more specific issues arise. Generally any LSEP issue that arises can be resolved in a way that does not harm the functionality of Camel, some are more creative than others but it is always helpful to consider how other similar systems have overcome similar issues.


\section{Software Development}

\addcontentsline{toc}{subsection}{Nature of CAMEL project}
\subsection*{Nature of CAMEL project}
The CAMEL project is basically a prototype e-learning system for mathematics that has been developed by Cardiff School of Mathematics. The existing codebases needs a complete overhaul: documentation, exception handling, unit test and logging are currently non-existent and there are several technical issues to resolve. The way in which students and teachers might interact with the system should also be investigated and the design would be modified accordingly.
 
	If the systems are implemented successfully, extensive data in student engagement will become available, which offers the possibility of analyzing patterns of student engagement and collaboration. The project should consider how such information might be extracted and visualized in a useful way. 

\addcontentsline{toc}{subsection}{Agile Model}
\subsection*{Agile Software Development}
First and foremost, manifesto for Agile Software Development has laid out four values which is:

\begin{enumerate}[label=(\roman*)]
\item Individuals and interaction over processes and tools
\item Working software over comprehensive documentation
\item Customer collaboration over contract negotiation
\item Responding to change over following a plan 
\end{enumerate}

These four values of Agile method are environment that we are currently facing while working with the CAMEL project since our team members need to work together in terms of understanding and breaking the existing code. Since the existing prototype needs to be fixed, a working software is what are we aimed for. Along the way as we overhaul the existing codebase, client involvement is really crucial every now and then, to ensure the prototype is according of the requirements. Besides, our client is working together through the process of overhauling the existing code. Thus, we will respond to change as if our client has any changes to made the requirements that we have agreed on.  

	Secondly, Agile has 12 Principles (Layton, Agile Project Management For Dummies) that are a set of guiding concepts that support teams in implementing agile projects. One of the concepts says the most efficient and effective method to convey information to and within a development team is face-to-face conversation. We have planned to meet up at least once in a week for our group to study the existing code. Furthermore, learning code would need discussion and expertise from each and every one of us. 

\addcontentsline{toc}{subsection}{The Agile Roadmap to Value (Deliverables plan)}
\subsection*{Next Phases of the Project}
Lorem ipsum dolor sit amet, consectetur adipiscing elit. Sed rutrum viverra nunc pellentesque finibus. Fusce id diam venenatis, tincidunt tellus sed, commodo sem.

\newpage
\begin{thebibliography}{99}
\addcontentsline{toc}{section}{References}

\bibitem{CDaPA} Copyright Designs and Patent Act (1988)
\newline
(\url{https://www.gov.uk/government/uploads/system/uploads/attachment_data/file/462194/Copyright_Designs_and_Patents_Act_1988.pdf})

\bibitem{LSEPI in Computer Tech} Legal, Social, Ethical and Professional Issues in the Application of Computer Technology 
\newline
(\url{http://www.macs.hw.ac.uk/macshome/MScComputing/RM/Docs/L4PLESI.pdf})

\bibitem{JiscLegal - Intellectual Prop.} JiscLegal on Intellectual Property
\newline
(\url{ http://www.jisclegal.ac.uk/legalareas/copyrightipr/ipressentials.aspx})

\bibitem{DPA:tDPP:S1} Data Protection Act: The Data Protection Principles - Schedule 1
\newline
(\url{http://www.legislation.gov.uk/ukpga/1998/29/schedule/1})

\bibitem{DPA:tDPP:S2} Data Protection Act: The Data Protection Principles - Schedule 2
\newline
(\url{http://www.legislation.gov.uk/ukpga/1998/29/schedule/2})

\bibitem{DPA:tDPP:S3} Data Protection Act: The Data Protection Principles - Schedule 3
\newline
(\url{http://www.legislation.gov.uk/ukpga/1998/29/schedule/3})

\end{thebibliography}

\end{document}
