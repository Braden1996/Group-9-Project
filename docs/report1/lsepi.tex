\newpage
\section{Legal, Social, Ethical and Professional Issues (LSEPI)}
    When considering the creation of a system like CAMEL it’s easy to get lost amongst the development process and overlook assessment of the legal, social, ethical and professional issues (LSEPI) raised by the creation and use of such a system.
\addcontentsline{toc}{subsection}{Intellectual Property}
\subsection*{Intellectual Property}
CAMEL is a system designed for use by academic establishments alone, liberating it from most relevant legal constraints since exceptions in the Copyright Designs and Patent Act (1988) are made for intellectual property used for educational purposes and typical uses are not considered public presentation in the context of copyright infringement.
\\\\If a lecturer wished to copy and via CAMEL share someone else's intellectual property, perhaps in the form of examples, diagrams or exercises, with his pupils he must within reason endure it is accompanied with sufficient \textbf{acknowledgement of the author}. Since this copy will be made available to others online, it is copyable and thus legally is an "\textit{accessible copy}" placing a responsibility on the lecturer that it is used \emph{only for the sake of education} (and if it is from another academic establishment, it must also be accompanied by a statement that the copy was made under the \cite[(Pt 1, Ch 3, §31B)]{CDaPA}).
\\If the work is made for the academic establishment it is its intellectual property and can be used freely within it, if it is used elsewhere, it must show acknowledgement to the academic establishment.
\\If a student submits something with enough originality to be deemed intellectual property, it must comply with the conditions stipulated in the Copyright Designs and Patent Act (1988) as stated above otherwise it can may be deemed as copyright infringement.
\\Taking this into consideration, CAMEL could have some sort of anti-screenshot/ anti-ctrl+x functionality to minimise the use of accessible copies of external works by pupils for non-educational purposes. As non-educational uses of Maths module material are next to nonexistent if existent at all, functionality could be incorporated to watermark screenshots with relevant acknowledgements or append it to strings copied with a browser’s “copy/paste” command.
\\If however, this intellectual property was used by a party other than the owner in any way but for education (especially if it would generate profit), it could infringe copyrights if either the owner is not acknowledged. This would affect the party’s right to paternity of their work/to be recognised as the owner, or the work was tampered with in some way that affected the work’s integrity (\cite{LSEPIinComputerTech}).
\\\\Legal action could be taken even in an ambiguous case if it acts against the spirit of the law. The owner could seek an injunction, interdict or other legal order to stop the violating party from making further infringing use of their work. The party can also impel or even oblige the violating party to remove or give up their copy of the infringing material and even claim damages from them (\cite{JiscLegal-IntellectualProp.}).
\\\\Even if a legal case cannot be made against ethically ambiguous use of another's intellectual property, efforts should be given to protect the general right to paternity and integrity of the owner as a gesture of goodwill.
\\This is especially important between and within establishments of education since we tread so dangerously along this line between ethical and unethical practice when concerning intellectual property, it is hard not to when dealing with so many new learners dealing with new technologies (\cite{JiscLegal-IntellectualProp.}). Since one member of an academic establishment is equally vulnerable to the bad faith of another, a two-way consideration benefits both parties in the long-run simply by maintaining a mutual respect between us all.
\addcontentsline{toc}{subsection}{Data Protection}
\subsection*{Data Protection}
Personal information will need to be stored in CAMEL's database so consideration must be taken of potential issues around its \textbf{processing} and \textbf{containment}, particularly relating to security. It is very important that CAMEL be reliably trustworthy with any personal information to assure users sensitive information is in good hands.\\\\
Currently it is decided that CAMEL will store information such as the identity of its users, the leaders and enrollees of modules, pupil test results and solutions, email addresses and any statistical analysis of student engagement.\\
Almost any personal information could be deemed sensitive. Not only is it subjective where on the spectrum personal information becomes sensitive, but the measures of sensitivity of information usually centre on the worst of the possible outcomes of leakage, which is ever changing with innovation and technology. Rather than dwell on what is sensitive and what is it not, it is best to treat all personal information, with the same care as sensitive information. Of course this does not concern data where informed legal consent has been given otherwise, or that has already been deliberately publicly shared by the individual.\\\\
From a socio-ethical standpoint, this means the database \emph{must} be kept \emph{incredibly} secure. Any means by which a party with no essential need to view or use the data can should be limited; only those with the required permissions should be able to see the data, and only those who are required to see the data should have those permissions. Anyone else should not be able to see this information under \emph{any} circumstances.
\\This emphasises the necessity of CAMEL’s need to be airtight in security, measures must be put in place to remove the possibility of any user accidentally stumbling upon (or intentionally) this information and should be able to resolve any issues that could result in a privacy breach, including glitches. These things are already expected of any professional system that stores personal information.\\
Even if we are certain our system security is airtight, we are ethically bound to minimise the loss and damage of personal information should a breach occur. This means that for any personal information, the necessity of its existence in the database must be evaluated against its sensitivity and the implications of its storage in the case of data leakage.\\\\
On the legal side of things, storage and processing of an individual’s data need only comply with the \textbf{DPA} (\textit{Data Protection Act 1998}). This means that in the context of our CAMEL system, any personal information about an individual can only be stored for processing if it is necessary for the performance of some contract the individual is part of, or if the individual gives consent to their information being used this way (\cite{DPA:tDPP:S1}, \cite{DPA:tDPP:S2}).
\vspace{0.35cm} \hrule {\raggedleft \scriptsize For example, since every University holds contracts with all of its pupils promising high level assessed education and an academic degree upon completion of each course, any reasonable data collection conducted to meet these terms need not consent from the pupil to be lawful.\par} \vspace{0.35cm} \hrule
If a personal information is disclosed but processing consent has not been accepted or denied, it may still be processed if out of legitimate interest by those it was disclosed to (\cite{DPA:tDPP:S2} -- unless protest is raised on grounds of potential resultant prejudice, i.e. a party suspects processing of this data will cause the individual in question to be unfairly prejudiced for reasons such as that of sex, learning difficulties, nationality, race, age etc.).
\vspace{0.35cm} \hrule {\raggedleft \scriptsize E.g. Once receiving their grades from a module exam, a subject leader need not ask consent from his pupils to process their results to analyse the performance of the subordinate teacher teaching them.\par} \vspace{0.35cm} \hrule
If this personal information is sensitive, storing and processing it is only lawful if \emph{explicit} informed consent has been given by the subject (or this data is already public by their deliberate actions \cite{DPA:tDPP:S3}). It is considered good legal (and ethical) practice to fully disclose the user on why and how their personal data will be processed, stored and used in an upfront, clear manner. It can legally be argued consent was not given explicitly enough otherwise.
\\This means that CAMEL can easily avoid legal disputes by adding into the "Terms and Conditions of Use" for any user, pupil or otherwise, that any information gathered/stored \& processed for the need of CAMEL’s functions (to be specified) has the explicit consent of the user. Though it should be noted that many users will skip over long T\&Cs, though this protects the establishment legally, on ethical grounds it is recommended that disclosure is repeated later.
\vspace{0.35cm} \hrule {\raggedleft \scriptsize For example, pupils could select an “opt-in” checkbox before a test to explicitly consent to timestamps, timing, results and their solutions being stored, processed for analytical reasons and used for educational purposes for future pupils -- either by being shown anonymously  to future classes, or via anonymous peer marking.\par} \vspace{0.35cm} \hrule
Any information stored \& processed must be \textbf{relevant and non-excessive} relative to the reason it was stored for to be legally sound, it must also be kept \textbf{accurate}, \textbf{updated} and \textbf{no longer than necessary}. There is an obligation for measures to be put in place to protect this personal data from “\textit{unauthorised or unlawful processing … [and] accidental loss\ldots destruction\ldots, or damage\ldots}” (\cite{DPA:tDPP:S1}).
\\This means that any personal data CAMEL stores in its database must either be regularly updated (and if applicable, eventually deleted) manually or automatically.
\vspace{0.35cm} \hrule {\raggedleft \scriptsize For example, CAMEL could include a regular updating and synchronisation process with the University's Intranet system. \par} \vspace{0.35cm} \hrule
It is also important that information that is not necessary for storage is not logged or saved after automatic deletion from the system’s cache.
\vspace{0.35cm} \hrule {\raggedleft \scriptsize E.g. CAMEL may check which browser a pupil is using if they click a "Having problems submitting?" link for debugging, this should be deleted once the system is informed the issue is resolved. \par} \vspace{0.35cm} \hrule
Recording users' IP addresses must comply with the DPA. A record of IP addresses used to log into a user's account could aid fraudulent practice investigations and security breach detection giving such a record just reason ethically and legally, but it could be argued such a log is an unnecessary risk once considering secondary uses for IP addresses such as determining sensitive information like location.
\\This could be argued to be a violation of the spirit of the DPA if sufficient security is not implemented or the information is kept longer than necessary (e.g. when the pupil has received a immutable mark).
\vspace{0.35cm} \hrule {\raggedleft \scriptsize E.g. recording multiple logins into a user’s account from separate locations/IPs could determine the existence of fraudulent practice by pupils or other users without the pupil’s knowledge, the system could then alert the user or an admin of a potential security issue and perhaps suggest a change of password.\par} \vspace{0.35cm} \hrule
\addcontentsline{toc}{subsection}{Analysis}
\subsection*{Analysis}
The analysis of data pertaining to an individual can be controversial; many can perceive it as an objectionable, invasive and insensitive systematic breach of privacy depending on how it is conducted and why, thus giving CAMEL’s analytical functions potential rise to social and ethical issues with regards specifically to data gathered about pupils.
\\\\One functionality the client hints at desiring is the ascertainment of work groups by considering similarities in the answers of pupils. On one hand, this could be perceived as socially unjust to pupils who feel they are being matched with others based on their weaknesses, but on the other hand if the same analysis was conducted for plagiarism detection there would be few objections.
\\It should be noted that any evidence of plagiarism gathered this could be largely attributed to coincidence, especially in Mathematics. Perhaps a compromise could be the inclusion of an opt-out option for pupils.
\\\\The results of a pupil seeing their assessment results relative to the rest of their class as part of the student engagement analytic of the system can be ambiguous. They could feel gratified and motivated, or if their efforts have not yielded proportionate success, their academic success could be inhibited by demotivation and lost confidence. It could be argued that it is a pupil's responsibility to learn to take valid criticism, but perhaps it is more ethical to make viewing such information unavailable to prevent negative consequence, or, in consideration of the potential benefits and in the spirit of full disclosure, make it optional via an opt-in button or feature.
%%%%%%%%%%%%%%%%%%%%%%%%%%%%%%%%%%%%%%%%%%%%%%%%%%%%
\\\\CAMEL will time test completion to aid monitoring student progress and plagiarism detection. The issue of showing this to the student can be argued similarly as above, however, since most Intranet systems do this without opposition, we will be too, but perhaps in future we could allow this information to be viewed optionally or remove the option all together should objections arise since pupils have access to the effectively the same information by timing themselves.
\addcontentsline{toc}{subsection}{Marking}
\subsection*{Marking}
Policies implemented when marking summative work must transfer from hardcopy marking to online marking via CAMEL to meet the same duty of care. One such policy is strict anonymity.
\\Any questions that must be marked manually should be presented to the marker without the identity of the student being revealed. While this is mandatory for summative assessments, teachers may wish to continue to avoid the bias associated with marking named work and use formative assessments as more valid practice by always hiding student identities when manually marking.
\\This could however limit the availability of tailored feedback, so perhaps anonymity could only be implemented summative work depending on whether the validity anonymity brings outweighs the extra benefit of named, better informed feedback or not.
\\Alternatively functionality could be implemented to reveal the student's identity after marking and before feedback is sent or at the manual request of the marker.
\\\\Student submissions will be processed through the system and viewed by various members of staff and possibly pupil peers depending on future functionality. Anonymity will not be kept in all of these instances, so students must be informed how their submissions can be used and that they may not always be anonymous (with specific stipulations) or, perhaps, an opt-in option can be proposed instead/also.
\\Our client has emphasised the importance of peer marking and its benefits to a student's critical analysis skills. It also helps aid a marker's workload with regards to formative submissions. If a student's anonymity cannot be guaranteed they may object in order to avoid peer scrutiny. Since little can be gained from sharing student identities in this case, anonymity should be maintained.
\\However students may still wish for their submission to remain entirely private, in which case perhaps an opt-out functionality can be implemented so the teaching staff must mark it instead.
\addcontentsline{toc}{subsection}{Conclusion}
\subsection*{Conclusion}
While this is a comprehensive discussion of the legal, social, ethical and professional issues regarding the proposed CAMEL system, but it should be noted this is not an exhaustive list and is mainly meant to serve the purpose of aiding further, more exhaustive discussion of CAMEL once more specific issues arise. Generally any LSEP issue that arises can be resolved in a way that does not harm the functionality of CAMEL, some are more creative than others but it is always helpful to consider how other similar systems have overcome similar issues.
