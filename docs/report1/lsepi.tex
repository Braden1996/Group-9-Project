\section{LSEPI}
    When considering the creation of a system like Camel it’s easy to get lost amongst the development process and overlook assessment of the legal, social, ethical and professional issues (LSEPI) raised by the creation and use of such a system.
    
    \addcontentsline{toc}{subsection}{Intellectual Property}
    \subsection*{Intellectual Property}
    Camel is a system designed for use by academic institutions alone, liberating it from many legal constraints.
        
    With regards to this system, we need only to focus on 2 kinds of intellectual property; that of parties outside of the educational establishment and that of parties within. Though in Maths, works like formulas aren't copyright protected, sufficiently original algorithms, images, video clips (etc.) are.

    If, for example, a lecturer wishes to copy and, via the system, communicate example questions, exercises or diagrams from external sources to other members within his educational establishment (namely his pupils), he must make this copy solely for the purposes of education/instruction for a noncommercial purpose and accompany it (unless impossible due to reasons of impracticality or otherwise) with sufficient acknowledgement of the author. Since this copy will be available to others in a copyable format it is legally classified as an “accessible copy” and comes with the responsibility of ensuring it is used only for educational purposes. If it is not an extract of works by an educational establishment, it must also be accompanied by a statement that the copy was made under the Copyright Designs and Patent Act (1988).\cite[(Pt 1, Ch 3, §31B)]{CDaPA}

    It should also be noted that since this work is presented to members within and connected directly to an educational establishment by a teacher/pupil in the course of activities of educational establishment for the purpose of education/instruction for a noncommercial purpose, the presentation of this work is not considered public in the context of copyright infringement.

    This could prompt Camel software developers to integrate some sort of anti-screenshot/ anti-ctrl+x functionality to minimise the use of accessible copies of external works by pupils for non-educational purposes, though it should be considered that most material that maths lecturers would share with pupils could not be used in any other way but educational so perhaps rather than attempting to prevent duplication of this material, functionality could be incorporated to watermark screenshots with relevant acknowledgements or append it to strings copied with a browser’s “copy/paste” command.

    Intellectual property (in the form of their own words, images, audio clips etc.) of staff within the academic establishment contributed as module content/material could either be treated with the same respects as external parties as an individual author or as part of the establishment with non-individual rights, in which case works would be acknowledged as property of their academic establishment, for example, ”© Copyright protected Cardiff University”. 

    Another example of a party within an academic establishment using Camel that may contribute their intellectual property for use via the system is any pupil who submits their own work perhaps as part of an assessment, for example. In this case any use of their work for the sake of education, if complying with the conditions stipulated in the Copyright Designs and Patent Act (1988) as stated above, similarly cannot be deemed as copyright infringement.

    If intellectual property of one of these parties was used in any way that was not in the name of education (especially for commercial purposes, particularly that of which would generate profit that does not go to the owner) without proper acknowledgement affecting the party’s right to paternity of their work (i.e the right to be recognised as the owner) or the work was tampered with in some way that affected the work’s integrity it would and thus infringe copyright law\cite{LSEPI in Computer Tech} if not by the letter of the law then at least in spirit of the law. In this situation the party whose intellectual property rights have been infringed can seek to take legal action in the form of an injunction, interdict or other legal order to stop the infringing party from continuing to violate their rights by making further infringing use of their work. The party can also impel or even oblige the infringing party to remove or give up their copy of the infringing material and even claim damages from them.\cite{JiscLegal - Intellectual Prop.}
    
    Even if the spirit of the law does not create enough of a case for the owner to press charges of infringement, it should be noted that efforts should be given to protect the general right to paternity and integrity of an individual’s intellectual property as a gesture of goodwill, especially between and within establishments of education since we tread so dangerously, with regards to intellectual property, along the line between ethical and unethical practice, it is hard not to when dealing with so many new learners dealing with new technologies,\cite{JiscLegal - Intellectual Prop.} and since any member of an academic establishment is equally vulnerable to the bad faith of other members of an academic establishment, a two-way consideration benefits both parties in the long-run simply by maintaining a mutual respect between us all.

    \addcontentsline{toc}{subsection}{Data Protection}
    \subsection*{Data Protection}
    In regards to Camel’s database, legal, ethical and social issues could potentially arise around the security, processing and containment of personal and sensitive information. Users will trust Camel to take care of their personal and sensitive data, particularly that of their name, test results, unique school IDs/emails, etcetera which will be processed mainly via storing and statistically analysing. The possible repercussions (and solutions) to this process must be considered.

    Since we are dealing with trust, ethically, what information is being stored and how it is used must be taken into consideration; how essential its existence in the database is to the function of Camel must be weighed against the sensitivity of the data and the potential implications of its storage, particularly in the case of its leakage.

    Examples of sensitive information could range from anything between a party’s name or address to their learning difficulty or financial situation to their national insurance number or passport number. Given any piece of information about a person, it can be difficult to determine whether it could be classified as sensitive or not, one could ask the person in question or refer to common census but more commonly we err on the side of caution and treat all personal and sensitive information with the same level of caution, aside in the context of requesting legal consent and in the instance where personal information is deliberately shared (e.g. on a professional networking site known as LinkedIn, all members are expected to go by their real names).
    
    On the legal side of things, storage and processing of an individual’s data need only comply with the DPA (Data Protection Act 1998), this means that in the context of our Camel system, any personal information about an individual can only be stored for processing if it is necessary either for the performance of some contract the individual is part of, or if the individual gives consent to their information being used this way.\cite{DPA:tDPP:S1}\cite{DPA:tDPP:S2}
    
    \vspace{0.35cm}
    \hrule
    {\raggedright \small \em “\ldots personal information about an individual can only be stored for processing if it is necessary \ldots for the performance of some contract the individual is part of…” \par} 
    {\raggedleft \scriptsize E.g. a university may have a contract with a student to award him an academic degree of a class relating to his performance over the duration of a course and would thus have the legal right to store and process examination results throughout his course in order to fulfil this contract to a high standard. \par}
    \vspace{0.35cm}
    \hrule

    The DPA also stipulates that if an individual’s personal information is disclosed but consent to processing has not yet been received, it may still be processed if out of a legitimate interest by those it was disclosed to unless it is unwanted in caution of potential prejudice of the individual’s rights, freedoms or legitimate interests\cite{DPA:tDPP:S2} (i.e. a party suspects processing of this data will cause the individual in question to be unfairly prejudiced for reasons such as that of sex, learning difficulties, nationality, race, age etc.).

	\vspace{0.35cm}
    \hrule
    {\raggedright \small \em “\ldots[disclosed personal data that has not yet received processing consent] may still be processed if out of a legitimate interest by those it was disclosed to\ldots”\par}
	{\raggedleft \scriptsize E.g. Once receiving their grades from a module exam, a subject leader need not ask consent from his pupils to process their results to analyse the performance of the subordinate teacher who taught them the module content.\par}
    \vspace{0.35cm}
    \hrule

    In addition, any personal information considered sensitive can only be stored and processed lawfully if either explicit consent has been given for the information to be stored and processed this way or this data is already public by deliberate actions of the individual in question.\cite{DPA:tDPP:S3} This means that Camel can easily avoid legal disputes by adding into the terms and conditions of use for any user, pupil or otherwise, that any information gathered for the need of Camel’s function has the consent of the user to be attained, stored and processed. Though it should be noted that many users will skip over long terms and conditions to continue with their work, though this protects the establishment legally, on ethical grounds it is recommended that disclosure is repeated when that information is gathered.

	\vspace{0.35cm}
    \hrule
    {\raggedleft \scriptsize For example, pupils could select an “opt-in” checkbox before a test to explicitly consent to timestamps, timing, results and solutions being stored, processed for analytical reasons and being used for educational purposes for future pupils either by being shown anonymously  to future classes or via anonymous peer marking.\par}
    \vspace{0.35cm}
    \hrule

    It is considered good practice (both in an ethical and legal sense) to fully disclose the user why and how their personal data will be processed, stored and used in an upfront, clear manner. 

    Any information/data stored/processed must be relevant and non-excessive relative to the reason it was stored for to be legally sound, it must also be kept accurate, up-to-date and not be kept for longer than necessary. There is also mention in the DPA of an obligation for measures to be put in place to protect this personal data from “unauthorised or unlawful processing … [and] accidental loss\ldots destruction\ldots, or damage\ldots”.\cite{DPA:tDPP:S1} This means that any personal data Camel stores in its database must either be manually updated regularly or done so automatically.

	\vspace{0.35cm}
    \hrule
    {\raggedleft \scriptsize For example, Camel could include a regular updating and synchronisation process with an academic establishments Intranet system. \par}
    \vspace{0.35cm}
    \hrule

    It is also important that information that is not necessary for storage is not logged or saved after automatic deletion from the system’s cache.

	\vspace{0.35cm}
    \hrule
    {\raggedleft \scriptsize  For example, we may find need to store reference to which browser a pupil is using in case submitting glitches due to  browser incompatibility, this should be deleted after the pupil either submits their work successfully or is given a disclaimer on the screen reading something like “Having problems submitting? Try again on Chrome, Internet Explorer or Opera”.\par}
    \vspace{0.35cm}
    \hrule

    Considering the ethical, social and professional side of it, If personal data is stored within a database, it must be kept secure. We contribute to this by minimising the potential problems arising from a privacy breach with damage control. Any means by which a party with no essential need to view or use the data can view or use the data should be limited; only those with the required permissions should be able to see the data, and only those who are required to see the data should have those permissions. Anyone else should not be able to see this information under any circumstances. This emphasises the necessity of Camel’s need to be secure, air-tight unhackable security can be strived for by putting in place measures to remove the possibility of any user accidentally stumbling upon (or intentionally) this information, in fact these security measures should be able to resolve any issues that could result in a privacy breach, including glitches.

    Recording a users IP address would also come under jurisdiction of the data protection act. There may be a desire to store a pupil's IP addresses on login in order to help determine whether there has been a security breach. Recording an IP could be argued to be unnecessary and a breach of privacy since the risks associated with storing a user’s IP include its potential use to determine some sensitive information like, for example, their location during use of the system, which could indicate their home address or a place of leisure they frequent. This means storing a user’s IP within a database could be a potential violation of the spirit of the DPA given sufficient security is not implemented, or the information is not deleted as soon as it is not necessary.

    Since storing IPs and timestamps could be used in the name of security it could serve to be a more vital function for Camel, justifying it ethically and legally.

	\vspace{0.35cm}
    \hrule
    {\raggedleft \scriptsize For example, catching multiple logins from separate locations/IPs into a user’s account could determine the existence of fraudulent practice by pupils or other users without the pupil’s permission, which would then allow the system to alert the user or an admin of a potential security issue and suggest a change of password.\par}
    \vspace{0.35cm}
    \hrule
    
    This functionally could be maintained in an ethical way if this data is stored only for a limited time before automatic deletion, perhaps until the pupil has received an official immutable mark for the work at the end of the year, module, semester or week, depending on the structure of the specific course.

    \addcontentsline{toc}{subsection}{Analysis}
    \subsection*{Analysis}
    The analysis of data pertaining to an individual can be controversial; many can perceive it as an objectionable invasive and insensitive systematic breach of privacy depending on how it is conducted and why, thus giving Camel’s analytical functions potential rise to social and ethical issues with regards specifically to data gathered about pupils.

    One functionality the client hints at desiring is the ascertainment of work groups by considering similarities in answers of pupils. On one hand, this could be perceived as socially unjust to pupils who feel they are being matched with others based on their weaknesses, but on the other hand if the same analysis was carried out for, and justified by, its contribution to detecting plagiarism it would raise next to no issues, but it should be noted that any evidence gathered in support of fraudulent submissions by pupils could be largely attributed to coincidence, especially in Mathematics. This weigh in against the advantages and disadvantages of such analysis prompts the inclusion of an opt-out option for pupils upon initial use of the system.

    Extending on this, a further social issue arises when a pupil is shown their results relative to the rest of their class. While some pupils may benefit from this by motivating them to excel further or by giving them gratification where it is due, some may be demotivated if their efforts have not yielded proportionate success potentially afflicting them detrimentally. This raises the question; is it ethical to show a pupil this being aware of the potential effect it could have? While it could be argued that its their own responsibility to manage their response to an accurate assessment of their progress, consideration could be given to those less able to manage that responsibility, particularly those who may not be underachieving relative to the rest of the class. This could be resolved by choosing not to show this to any pupil at all, in consideration of the possible benefits to a pupil’s success and in the spirit of full disclosure, it would probably be optimal to show these results only to those who wish to see it by for example requiring them to click a “show my results against the rest of the class” button first.

    Camel will time how long it takes students to complete assessments, this is necessary both in detecting plagiarism and in assessing the ability of the pupil, as to whether a student should be shown this information is somewhat debateable for reasons similar to that regarding their results relative to the rest of their class. However since most Intranet systems with online tests do this without any objection from staff or students, we will be implementing this regardless though it should be noted that in future we could allow this information to be viewed optionally very easily, or even remove the option all together since a pupil can effectively access the same information by timing themselves. These options can be explored should any issues arise in the future.

    \addcontentsline{toc}{subsection}{Marking}
    \subsection*{Marking}
    It has always been an issue of professionalism when unintentional bias by a teacher could affect the mark of a student, since the class of one's’ degree is so important in their migration to the professional world it is important Camel maintains the same cautionary measures other forms of course submission and marking do.

    Hardcopy assessment submissions are assessed as anonymous submissions by lecturers in order to meet a standard of care for a lecturer’s pupils. This should be mimicked in the functionality of Camel where any answers that must be manually marked by a member of staff should be done so without the identity of the pupil in question being revealed. It is best to do this with both formative and summative assessments but it should be acknowledged that this could potentially limit the ability of the lecturer to provide feedback tailored to the student since any helpful understanding of the pupil that the teacher may have is made redundant and also, such obtaining such understanding from the test results is inhibited by anonymity. This could be worked around if for example, anonymity was only maintained for summative assessment and not formative assessment which is, after all, primarily meant to provide practice for, aid the teacher in the education of, and give the teacher opportunity to give constructive feedback to their student. Alternatively the teacher could mark each question with the student’s identity hidden, only to be revealed automatically once the mark has been given or perhaps manually only in the instance the teacher feels it prudent to give individual feedback.

    Although it may seem abundantly clear, some students may think that being assessed through an online test means that their answers will be automatically marked and viewed by no one when in reality, answers that must be manually marked or those that a lecturer may choose to show as an example for future classes will be viewed by other people. In this case not only must all anonymity be maintained (unless perhaps in the latter case, a student wishes to be credited in which case perhaps an opt-in/opt-out button for anonymity can be included), but pupils must receive a disclaimer that for any open ended question, it may be viewed by a member of staff or by future students (though again, for the latter case, this is not a necessity and can be optional).

    An educationally beneficial means of marking (formative) assessment is peer marking, it teaches students to be critical of work thus giving them the ability to be critical of their own work should they try, and it lessens pressure and time constraints on a lecturer freeing his efforts up for meeting the educational needs of his students more effectively. This is something that students may object to, especially if their anonymity cannot be guaranteed, if they do not wish to be scrutinised by their peers. One way around this is to disclose to the student, before a test, that their results may be marked by their peers and perhaps even include an opt-out button so that their test will be marked by a teacher instead. It is also important that security measures be put in place to keep the students identity anonymous.

    \addcontentsline{toc}{subsection}{Conclusion}
    \subsection*{Conclusion}
    While this is a comprehensive discussion of the legal, social, ethical and professional issues regarding the proposed Camel system, but it should be noted this is not an exhaustive list and is mainly meant to serve the purpose of aiding further, more exhaustive discussion of Camel once more specific issues arise. Generally any LSEP issue that arises can be resolved in a way that does not harm the functionality of Camel, some are more creative than others but it is always helpful to consider how other similar systems have overcome similar issues.
