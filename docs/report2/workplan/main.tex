\section{Work Plan}
In this section, we will be explaining the work plan we made for this second phase of project and also upcoming final phase of the project. The Gantt charts will be the main guide for us completing our milestones and deliverables. Apart from that, we also made weekly work plans to keep track in more details of task that need to be completed on a weekly basis.

\subsection*{Second Report Gantt Chart Explained}

\addcontentsline{toc}{subsection}{Gantt Charts}
The Gantt chart for the second project is attached in Appendix 4. Since we have failed to produce the second phase of the Gantt chart in the previous report, we have drafted a provisional plan for every team member to be clear of our individual or sub-group task in 7/11/2015 (Week 11). It is essential to get the plan done early so that we can start working on individual fragments in reasonable periods before it is compiled as a whole. This second phase of our group project is scheduled to start on Week 10 in the Autumn semester until Week 3 in the Spring semester.\\

 From Week 10 until Week 12, we started to compile a list of bugs on the CAMEL system, began to fix bugs, amend database structure, have a concise design of the improved system, complete partial , produce Use Case diagram, provide test cases and also test it by unit. Then, we split tasks among us to kick off the second phase of the project. Although we originally have splitted the task in Week 11, we were distracted with other courseworks and exam commitment until Week 12. Thus, we finalised our task at the beginning of Week 1 in Spring semester and began to start to work on second phase of CAMEL development.   Apart from that, we also met our client to get feedback of the first phase of the project. Starting in the Spring Semester, from the first until the third week, we continued towards the development of the CAMEL system. We scraped the system structure then came out with a refreshed software architecture and data design that we found that would work better for CAMEL.\\
 
	We planned to allocate spending more time on making progress towards solving the software development problem, especially on adjusting software architecture and data design of the system. Then, halfway through the development, we further into partial implementation, testing and provide test cases of the latest prototype that we have been working on at the moment. Next, Use Cases diagram is plan to develop after the latest system working so that we can show how the interface works in this Interim Report.\\
	
Last sprint of the second phase of the project is to ensure that we plan to justify evidence of work accordingly and considered some legal, social, ethical and political issues so that we do not go against any related laws. We are aware that our plan/product roadmap needs to be revised in accordance with the  situation we faced as a group.  Since we are following Agile Software Development Management, one of the Agile Principles that we prioritised is working software, which is the primary progress of the measure.\\

*Disclaimer: The Gantt chart looks smaller when it is compressed to fit in A4 because the only alternative we found at the moment was to export the Gantt chart is via Smartsheet.com.

\subsection*{Final Report Gantt Chart Explained}

The Gantt chart for the final report is attached in Appendix 5. We are scheduled to start development for our third report at the start of Week 4, after report 2 has been submitted. We have allocated from Week 3 to Week 9  to complete all physical relevant requirements within our software. As our project is heavily dependent on fixing and amending existing code, we feel allocating a large time to this will help us all complete and implement the required requirements, hence the long period seen on the Gantt chart for the first few milestones.\\

From the Week 10 to the end of Week 11, we aim to complete the development and compiling of the third report, having completed all the software implementations, we can then develop justifications of our actions and produce a write up. We will focus heavily upon evaluating the system in the final report and hope to achieve completing as many feasible requirements from our client’s specification. The reason that all the requirements Our main priority during this final report is to ensure the basics of the software are working as they should, and that the system’s core functionalities are met.\\

Within the weeks of developing the third report, we will also produce a presentation of our working software. We will include what we have developed, how we met the client's requirements and for any we were unable to meet we will justify why this was undurable during our project cycle .We will also show our working software and we will then be prepared to present this in week 12. \\

Relating to our original plan in report 1, report 3 will aim to cover the requirements of maintainable code, admin and developer documentation where relevant, however most user documentation will minimal as we aim to make the use of the CAMEL website to be self explanatory. It will also include testing frameworks, comparisons of users answers, PEP8 standardisation, authors page, methods of checking and comparing students answers, deadlines for students assignment submissions, functionality for the student to only see the modules they are enrolled on, ease of use will be consistent throughout the software and it will have security permissions to prevent unauthorised access. All stakeholders requirements and needs will be completed by report 3 unless requirements are changed.\\


\subsection*{Weekly Work Plan}
\addcontentsline{toc}{subsection}{Weekly Work Plan}

In order to manage our group further, we discovered that a clear concise weekly work plan (See Appendices 3 and 4) with internal milestones was a easier method to keep on track of minor milestones within the project. Each member had access to a copy of the work plan and it was confirmed which tasks needed to be completed before another task began, or what could run in parallel with the Gantt Charts activities. We have attached copies of the weekly work plans for report 2 and report 3 in the appendices. They allow us to monitor meetings with the group and the client to ensure we are consistent with the delivery of our software. They also show the availability of the group as a whole for discussions of the software we are re-creating. \\

We decided to include this approach as it showed how we managed our time, and how we plan to manage workloads during extensive periods such as the Easter break. It was also a more visually clear approach to view up and coming deadlines, and as an informal approach, a reminder as to any future  group meetings or client meetings we have planned. Apart from that, we think this is more accurate way  to communicate within our group. \\

\subsection*{Feasibility of Work Plan}

We all aim to follow the work plan accordingly, to achieve the projected results. By ensuring internal milestones are met we can ensure each task is continued thereafter, as some elements such as fixing all the current code in place needs to be completed before amending the code, to adhere to the requirements. We need to ensure the CAMEL system is in working state and we all know what the current code does, before we alter existing code and implement features to increase the usability for stakeholders. If we don’t fix the current system first, it will be much more difficult to implement further requirements as we may not have a full understanding of the client's current code in place, which will also make it a lot more difficult to resolve any bugs. \\

So far the Gantt chart has been feasible and we have adhered to the deadlines and milestones that were set from the first Gantt chart development. We have however amended the Gantt chart to meet to needs of future requirements, by setting milestones of tasks that need to be completed for both report 2 and report 3. We also amended the Gantt chart for report 1 as requirements such as Provide functionality of enrolled modules for students and Fix bugs with current system were dropped. By ensuring deadlines in the work plan can be met, it can be used as a tool to ensure each task is kept on top of. By adding additional features to our Gantt chart for future reports we can ensure the feasibility of our work plan is continued by viewing future deadlines, and from the experience of report 1 we are on target to meet all the set milestones to complete our project in due course. \\

We decided to use both the Gantt Chart and Weekly work plans as methods of portraying future developments.The Gantt Chart is also in place to aid us in monitoring the workload of each member, to ensure each member has a task to complete and is contributing to the project fairly, and the weekly work plans helps the with the organisation of the group and gives a brief outline of when internal milestones should be completed. \\

\subsection*{Risk Assessment and Risk Mitigation}
\addcontentsline{toc}{subsection}{Risk Assessment and Risk Mitigation}

We would also aid the justification of the feasibility of the  work plan by assessing the risks within our project by including a risk map. We identified the risks that could occur in the duration of our project and included the strategies we have in place should they occur. We included the calculations of the total risk by measuring the likelihood of the event occurring, against the impact it would have should it occur. The strategies in place helps us to maintain the project goals and control the overall risk. We followed the colour schemes of the risk map below so it is more visually appealing and clearer to assess how the controls we implemented help toward risk mitigation. \\

			\definecolor{greenI}{RGB}{226, 239, 217}
			\definecolor{greenII}{RGB}{197, 224, 179}
			\definecolor{greenIII}{RGB}{0, 176, 80}
			
			\definecolor{yellowI}{RGB}{255, 229, 153}
			\definecolor{yellowII}{RGB}{255, 255, 0}
			
			\definecolor{orangeI}{RGB}{244, 176, 131}
			\definecolor{orangeII}{RGB}{255, 165, 0}
			
			\definecolor{redI}{RGB}{255, 150, 150}
			\definecolor{redII}{RGB}{255, 0, 0}
            
            \begin{table}[!htb]
					\small
                \renewcommand{\arraystretch}{1.1}
					\begin{center}
						\caption{Risk Assessment Key}
						\label{table:riskassessmentkey}
						\setlength{\aboverulesep}{0pt}
						\setlength{\belowrulesep}{0pt}
						\setlength{\extrarowheight}{.75ex}
                        
                        \scalebox{1.11}{
                        \hspace{-1.9cm}
						\begin{tabular}{|p{3.5cm}|*{5}{p{2.5cm}|}}
							\toprule
							Total Risk & \multicolumn{5}{c|}{Impact} \\
							\midrule
							
							Likelihood of Happening & \cellcolor{greenI}1.Insignificant {\scriptsize (Minor problem easily handled)} & \cellcolor{greenII}2.Minor {\scriptsize (Some disruption possible)} & \cellcolor{yellowI}3.Moderate {\scriptsize (Significant time/resources required)} & \cellcolor{orangeI}4.Major {\scriptsize (Project severely damaged)} & \cellcolor{redI}5.Catastrophic {\scriptsize (Project ruined)}\\
							\midrule
							
							\cellcolor{greenI}1.Highly Unlikely & \cellcolor{greenIII}1 Low & \cellcolor{greenIII}2 Low & \cellcolor{greenIII}3 Low & \cellcolor{yellowII}4 Moderate & \cellcolor{yellowII}5 Moderate\\
							\midrule
							
							\cellcolor{greenII} 2.Unlikely & \cellcolor{greenIII}2 Low & \cellcolor{yellowII}4 Low & \cellcolor{yellowII}6 Moderate & \cellcolor{orangeII} 8 High & \cellcolor{orangeII}10 High\\
							\midrule
							
							\cellcolor{yellowI}3.Moderate & \cellcolor{greenIII}3 Low & \cellcolor{yellowII}6 Moderate & \cellcolor{orangeII}9 High & \cellcolor{redII}12 Extreme & \cellcolor{redII}15 Extreme\\
							\midrule
							
							\cellcolor{orangeI}4.Likely & \cellcolor{yellowII}4 Moderate & \cellcolor{orangeII}8 High & \cellcolor{redII}12 Extreme & \cellcolor{redII}16 Extreme & \cellcolor{redII}20 Extreme\\
							\midrule
							
							\cellcolor{redI}5.Almost Certain & \cellcolor{yellowII}5 Moderate & \cellcolor{orangeII}10 High & \cellcolor{redII}15 Extreme & \cellcolor{redII}20 Extreme & \cellcolor{redII}25 Extreme\\
							\bottomrule
						\end{tabular}}
					\end{center}
				\end{table}
The use of risk mitigation can be used for minimising risks. In our first report we included an in depth risk map that considered all possible risks that could occur during the duration of our project. We included controls to manage the risk in report 1 but we now also have strategies in place to minimise the effects should they still happen to occur. By lowering the risk initially with controls in place, this allows our strategies to all be feasible should an event happen, maintaining and preserving our software we are re developing. We have created contingencies for the risks as follows in the below chart: \\   

 
\subsubsection*{Risk Assessment Table}
            	\label{table:riskassessment}
				\setlength{\aboverulesep}{0pt}
				\setlength{\belowrulesep}{0pt}
				\setlength{\extrarowheight}{.75ex}
                \setlength\LTleft{-0.74in}
				{\tiny
				\begin{longtable}{{|p{2.5cm}|p{1.5cm}|p{7cm}|p{1.5cm}|p{6cm}|}}
					
					\caption{Risk Assessment} \\
					\toprule
					Risk  & Total Risk \scriptsize (Likelihood * Impact) & Controls & Total Risk Given Control \scriptsize (Likelihood * Impact) & Strategy for Risk Mitigation\\
					\midrule
					\endfirsthead
					\multicolumn{5}{c}%
					{\tablename\ \thetable\ -- \textit{Continued from previous page}} \\
					\midrule
					Risk  & Total Risk \scriptsize (Likelihood * Impact) & Controls & Total Risk Given Control \scriptsize (Likelihood * Impact) & Strategy for Risk Mitigation\\
					\midrule
					\endhead
					\hline \multicolumn{5}{r}{\textit{Continued on next page}} \\
					\endfoot
					\midrule
					\endlastfoot
					
					Short-term Illness of team member & \cellcolor{yellowII} 6 (3 * 2) & Regular de-briefing with group so all know what each member has done and will do.
					Write comments on major updates onto GitHub when committing.
					If it does happen, tasks for that member will be fairly re-assigned to other members. & \cellcolor{greenIII}3 (3 * 1) & If it does happen, tasks for that member will be fairly re-assigned to other members.  Contact can be made to the team member to see when they may return to assign another member who has a lower workload to take their tasks.\\
					\midrule
					
					Team member leaving the group & \cellcolor{yellowII} 5 (1 * 5 ) & Regular de-briefing with group so all know what each member has done and will do. & \cellcolor{yellowII} 4 (1 * 4) & If it does happen, tasks for the missing member will be fairly re-assigned to other members. Otherwise, we could communicate with supervisor and lecturer for alternatives: e.g. drop some requirements.\\
					\midrule
					
					Loss of work (eg. due to corrupt files or forgetting to save) & \cellcolor{orangeII} 9 (3 * 3) & Use of GitHub/Google Drive/auto-saving software. Everyone will save their work at least every 20 minutes.
					Routine backups will be made every week.
					Create a Gantt chart specifying every task to be completed and follow it.  & \cellcolor{yellowII} 4 (2 * 2) & If it does happen we will find the latest version of the work and schedule in the nearest available time to re-do the missing work. We will also reload the latest version of work as the majority of members have the system backed up onto multiple machines. Any individual work is recommended to be uploaded to group’s google drive. For interrupted work, we will keep all our code on our own github for the purpose of fulfilling our project and fork it to the clients repository on GitHUb as a whole at the end when the project is complete.\\
					\midrule
					
					Overrun on a task and miss a milestone & \cellcolor{redII} 12 (3 * 4) & Thorough planning of each task and appropriate deadlines with regular de-briefing with group so all know what is on schedule. Create a gantt chart specifying every task to be completed and follow it. & \cellcolor{yellowII} 6 (2 * 3) & If it does happen we will schedule in the nearest available time to complete the given task, and more group members can help the individual or sub team who have overrun. We will submit the current work we have if we are close to a deadline and feel we aren't able to meet it.\\
					\midrule
					
					Not all functional requirements are met by the deadline. & \cellcolor{redII} 15 (3 * 5) & Agree reasonable functional requirements with client. 
					Plan time well.
					Use of agile model could allow for adjustments of requirements.
					
					\textbf{If it does happen nothing can be done!} & \cellcolor{yellowII} 5(1 * 5) & We will make an internal deadline in order to complete all requirements as possible, and as a contingency submit what we have.\\
					\midrule
					
					Client indisposed & \cellcolor{yellowII} 6 (3 * 2) & Fully understand task and fully document requirements as and when they’re agreed. & \cellcolor{greenIII} 3 (3 * 1) & If it does happen we should ensure we are regularly having meetings with the client and try to resolve any issues we may have as soon as possible.\\
					\midrule
					
					Team member prolonged absence & \cellcolor{greenIII}  3 (1 * 3) & Regular de-briefing with group so all know what each member has done and will do. & \cellcolor{greenIII} 2 (1 * 2) & If it does happen, tasks for the missing member will be fairly re-assigned to other members. Otherwise, we could communicate with supervisor and lecturer for alternative: e.g. drop some requirements.\\
					\midrule
					
					Server failure (e.g., GitHub goes down, files incompatible with server) & \cellcolor{yellowII} 4 (1 * 4) & Routine backups will be made every week on hardware (eg USB)  and other alternative software. & \cellcolor{greenIII} 3 (1 * 2) & If it does happen we will find the latest version of the code that is backed up on another device, either from hardware or software and fill in any gaps that are missing at the earliest convenience.\\
					\midrule
					
					Django gets updated & \cellcolor{greenIII} 3 (1 * 3) & Make sure code is well-structured and documented. & \cellcolor{greenIII} 2 (1 * 2) & If it does happen we will continue to use the latest version we have been using and schedule in the nearest available time to complete work. \\
					\midrule
					
					Code corrupted, unusable or deleted & \cellcolor{redII} 12 (3 * 4) & Use of GitHub/Google Drive/auto-saving software.
					Everyone will save their work at least every 20 minutes.
					Routine backups will be made every week. & \cellcolor{yellowII} 4 (1 * 4) & If it does happen we will find the latest version of the work and continue from there again. Then, we will work overtime until we produce working code.\\
					\midrule
					
					Last minute bugs, 1 hour before deadline & \cellcolor{redII} 15 (3 * 5) & Have multiple people do thorough debugging on different systems. & \cellcolor{yellowII} 5 (1 * 5) & If it does happen we will ensure we work overtime to ensure there are no final bugs before submission. If any are found they should be fairly easy to resolve as our code will be organised.\\
					\bottomrule
				\end{longtable}	}


