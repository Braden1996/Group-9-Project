\subsection*{Revised Requirements}
\addcontentsline{toc}{subsection}{Revised Requirements}
	For the development of CAMEL, we have decided to revise some of the existing requirements that were proposed in the first report. We have decided to change a few of the requirements, as we have decided to go with a different approach to develop CAMEL than we originally intended in the first report. As a group, we have decided to remake the CAMEL system from scratch to allow natural implementation of many of the requirements, such as error handling, Python3 and maintainable code. By applying a complete code overhaul, we believe we can provide a better system in terms of quality and efficiency, rather than using the existing code base.\\    
	
	\subsubsection*{Functional Requirements}
	\underline{\textbf{New Functional Requirements}}
	\begin{itemize}
		\item Provide functionality for lecturer to add modules: We have decided to add some functionality for the lecturer as with the current state it requires a lecturer to go into the database. For security reasons, we do not want a lecturer to have this sort of access or risk the possibility of have data tampered with. With the current time scale, we believe that this sort of form can be easily implemented within a day and will not affect other requirements. 
		
		\item Provide functionality for lecturer to add latex books to CAMEL: From looking at our original requirements, we can to the conclusion that this requirement was missing from the original design, as a means to add files is required on the site. With the current aim at the moment is the develop a new parser. This could be limited in functionality due to time if the parser has any possible bugs and a way is needed to pull pictures from the files, but if successfully implemented it could be a nice addition.
	\end{itemize}
	
	\underline{\textbf{Continuing Functional Requirements}}
	\begin{itemize}
		\item Provide Error handling and exception handling: We have continued and considered this requirement from the ground up with the system that we are currently developing. With the code that has been written so far, error handling has already been applied to ensure CAMEL can handle any possible errors. The error handling has been naturally applied to the system, and we are on track to ensure the system covers all possible errors. 
		
		\item Provide Unit Tests of code: Unit tests are currently in the stage of being implemented soon, and will also occur naturally as we develop the CAMEL system with additional functionality. 
		
		\item Add methods to compare student answers: We have decided to continue to support this requirement as we believe that many of the changes that we are making to CAMEL will allow for easy integration of additional functionality, such as this. Development on comparing student answers will start as soon as the refactoring of the code is completed.  
		
		\item Logging of student answers: With the new code base under development, this functionality will be also be implemented as we believe that this can be easily integrated when developing the homework application. This functionality will be added when development on the homework application begins.  
		
		\item Include deadlines in tests: Another piece of functionality that we will be carrying on with. This will be implemented and factored in when the homework application is being implemented. The deadline will be displayed on screen and if a student does not complete the test, a message will be included.  
	\end{itemize}
	
	\underline{\textbf{Dropped Functional Requirements}}
	\begin{itemize}
		\item Provide functionality of enrolled modules for students: We decided to drop this requirement as it would only be a proof of concept rather than something that we can included into the system. We have decided to focus on items that we can implement rather than have development time spent on items that cannot be fully implemented.  
	\end{itemize}
		
	\subsubsection*{Non-Functional Requirements}
	\underline{\textbf{New Non-Functional Requirements}}
	\begin{itemize}
		
		\item New parser implementation: One of the main additions that we decided to tackle with the next build for CAMEL was the introduction of a new parser. We decided that a new parser had to be implemented as there was a few flaws with the code and it was generally very hard to work with. In our current build, a new parser has been included that allows for new latex markup to be introduced relatively quickly, while keeping the code maintainable which was not present in the original build.    
		
		\item Re-factoring of existing codebase: To allow for many of the requirements above to be easily introduced, we have decided to re-write much of the existing code to improve the quality and efficiency of it. By having all new code written, it also allows the integrations of the new parser to be implemented without causing compatibility issues. In the current build of CAMEL, you can see the newly introduced code meets many of the requirements, such as Python3, error handling and maintainability. 
	\end{itemize}
	
	\underline{\textbf{Continuing Non-Functional Requirements}}
	\begin{itemize}
		
		\item Documentation for developer: Documentation for the system will be included as originality intended to allow future developers of CAMEL to understand what core code does. We are assuming that this documentation will be available online with CAMEL, so we will be utilising a Django module to document all code as it is completed.
		
		\item System permissions: The use of system permissions can be introduced relatively quickly with Django's pre-existing user permissions that can easily allow for separation of lecturer and student areas. We are assuming that the main users are a student and lecturer who will not have access to the database. This will be handled by a third user called 'admin' that will overlook all data. 
		
		\item Maintainable code: This requirement is in the process of being done as we re-write CAMEL. To date the code that has been implemented has been written to keep this requirement in mind and conform to the PEP8 standard.
		
		\item Breakdown of core application: A requirement that has already been met and was successfully implemented. The new code base currently has the core application separated into many cohesive applications that do specific tasks. This can be seen in the code that we have attached.
		
		\item Python3 Upgrade: Python3 has been successfully tested and can be implemented throughout the system. One of the main motivations for the overhaul of the code was to ensure that Python3 can be rolled out as the original system had some differences that prevented Python3 from being introduced. Such as some of the libraries having different function names with the newer version of Python.    
		
		\item Improve navigation: Another requirement that we have met to date. Navigation has been resolved by splitting the core 'url' file down, as the large amount of look up queries was hindering performance.
	\end{itemize}
	
	\underline{\textbf{Dropped Non-Functional Requirements}}
	\begin{itemize}
		\item Fix bugs with current system: The introduction of overhauling the existing system has led to this being dropped as we intend to re-implement code and bug test each part as it is introduced. The current system that is being developed will have it's own bugs that are not part of the original code base.   
	\end{itemize}
	