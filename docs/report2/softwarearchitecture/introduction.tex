\subsection*{Revised Requirements}
\addcontentsline{toc}{subsection}{Revised Requirements}
	For the development of CAMEL, we have decided to revise some of the existing requirements that were proposed in the first report. We have decided to change a few of the requirements, as we have decided to go with a different approach to develop CAMEL than we originally intended in the first report. As a group, we have decided to remake the CAMEL system from scratch to allow natural implementation of many of the requirements, such as error handling, Python3 and maintainable code. We chose to do this, as the current build of CAMEL would have reduced code quality and efficiency, as the code was not in a state that could be used. By applying a complete code overhaul, we believe we can provide a better system in terms of quality and efficiency, rather than using the existing code base. The original requirements that we proposed can be found in our first report. \\    
	
	\subsubsection*{Functional Requirements}
	\underline{\textbf{New Functional Requirements}}
	\begin{itemize}
		\item Provide functionality for lecturer to add modules: We have decided to add some functionality for the lecturer as currently it requires the lecturer to go into the database. For security reasons, we do not want a lecturer to have this sort of access, as there is the possibility of have data tampered with as well a breach of data protection. With the current time scale, we believe that this sort of form can be easily implemented within a day and will not affect other requirements. 
	\end{itemize}
	
	\underline{\textbf{Dropped Functional Requirements}}
	\begin{itemize}
		\item Provide functionality of enrolled modules for students: We decided to drop this requirement as it would only be a proof of concept rather than something that we can included into the system. We have decided to focus on items that we can implement rather than have development time spent on items that cannot be fully implemented. Our main focuses now is creating functionality that can be used, and we feel that this is something that can be added in the next version of CAMEL.  
	\end{itemize}
			
\subsubsection*{Non-Functional Requirements}
	\underline{\textbf{New Non-Functional Requirements}}
	\begin{itemize}
		
		\item New parser implementation: One of the main additions that we decided to tackle with the next build for CAMEL, is the introduction of a new parser. We decided that a new parser had to be implemented, as there was a few flaws with the current parser code, and it was generally very hard to work with. In our current build, a new parser has been included that allows for new latex markup to be introduced quickly, while keeping the code maintainable which was not present in the original build. The introduction of a parser has allowed code quality to be maintained, which was not present before.   
		
		\item Re-factoring of existing codebase: To allow for many of the requirements that were proposed, we have decided to re-write much of the existing code to improve the quality and efficiency of it. By introducing new code, it also allows the integrations of the new parser to be implemented without causing compatibility issues. As well as providing a much better code base for future developers, it also makes expansion to CAMEL much simpler as everything is placed in a specialised file. In the current build of CAMEL, you can see the newly introduced code meets many of the requirements, such as Python3, error handling and maintainability. 
	\end{itemize}
	
	\underline{\textbf{Dropped Non-Functional Requirements}}
	\begin{itemize}
		\item Fix bugs with current system: The overhauling of the current system, meant that much of the current issues with CAMEL will be resolved with newer high quality code. As we are not explicitly fixing bugs on the original code base, this requirement is dropped as it was aimed at fixing only them. By having a new system means that we will have a new set of bugs to deal with that are not related to the original CAMEL system.
	\end{itemize}

	\subsection*{Updated Legal, Social, Ethical and Professional Considerations}
	This section contains an updated outlook on legal, social, ethical and professional issues associated with CAMEL. It is not exhaustive, but provides a space to explore issues that were not mentioned elsewhere in the document.

	\begin{itemize}
		\item Analysing student answers (lexical comparison to measure collaboration and engagement): Lexical analysis of student submitted answers can be reliable, consistent and appropriately indiscriminate and thus can provide a less socially objectionable system for plagiarism detection.
			However the validity of information yielded from such a system is questionable. This is especially true within subjects of science, where answers of an explanatory or computational nature may appear very similar yet actually be suitably original. This brings into question the credibility of this system. Though since all such software carries the same uncertainty, it should not affect the professional integrity of CAMEL.
			Any teachers that are misled to rely too heavily on such a system may treat a student unfairly if wrongly accused. As long as teachers are well informed that results are not definitive and should only be used as an indication of potential plagiarism, this should not be of much social concern. It is also prudent to remember that providing such a disclaimer and keeping other professionals properly informed is a professional obligation we must adhere to.

		\item Logging student answers: Since this information is not classed as personal information, storage of student’s answers should not breach the Data Protection Act. The exception to this is the attachment of the student’s identity to the piece of work. In order to comply with the spirit and word of the law, as well as common ethical practice, access to this information should be restricted only to those who need it and the information itself should not be kept longer than necessary.
			Students may wish to remain anonymous. Refusing this option could lead to socially based objection. It would also take control away from the student, hurting the system’s usability. This would of course in turn damage the professional success of CAMEL.
			A functionality of anonymity must be implemented anyway, since marking cannot be implemented fairly without it. In fact, non-anonymous marking would conflict with protocol for most academic establishments within our target market. This would reduce CAMELs software quality from a professional aspect.

		\item Allowing deadlines on student assignments to prevent late submissions: Since this may decide the fate of any given student’s final mark, the presence of any errors in this process could cause users significant issues.
			Though this is a social consideration, it should be noted that an error like this could drive users away and hurt the professional integrity of CAMEL.
			A potential work around could be a period of grace after a deadline. This could allow students time to rectify the problem or find an alternative. Unfortunately this method will not discriminate those who submitted late through their own fault. This is unfair to students who adhered to the deadline as they should have.
			An alternative is the allowance of late submissions that are then flagged for individual consideration later.
	\end{itemize}
