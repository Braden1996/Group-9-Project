\subsection*{Algorithms}
\addcontentsline{toc}{subsection}{Algorithm Design}
	\subsubsection*{Introduction}
		As expected during the development of any large system, we must produce solutions towards the many proposed sub-problems in which are collectively required for the entirety system to function correctly.
		
		Although the thinking required for some of these problems is rather trivial, that is not the case for every problem identified during the development of our system... In fact, we encountered several problems which required comprehensive study, along with thinking in conscientious manner, in order to formulate a robust solution.
		
		Throughout the remainder of this subsection, we shall discuss the problems which we encountered, along with our solutions to said problems.
		
		To create the opportunity for us to more efficiently spend our time on the system, we have not include pseudocode but instead included parts of our current Python code due to its high readability and similarities to the English language.
		
	\subsubsection*{\LaTeX Parser}
		As we have discussed previously (\fullref{sssec:latexparser}), our \LaTeX parser is comparatively complex with regards to the system's other components. Thus, we shall describe and break down the reasoning behind how our algorithms work. Due to our parser's logic being split into several components, we have accordingly categorised our reasoning into their respective components.
		
		It is advised that the reader be aware of the priorly referenced section; as our algorithms will build from those design choices.
		
	\subsubsection*{\LaTeX Parser's \TeX BookParser} \label{sssec:texbookparser}
		Firstly, we shall begin to describe how the logic behind our \TeX BookParser's \textit{parse\_latex} method.
		
		The function of \textit{parse\_latex} is to take the contents bounded between the \verb\begin{document}\ ... \verb\end{document}\ commands as input. The output is a list of \textit{Nodes} representing all the immediate children of the given \LaTeX document. Each of these immediate \textit{Nodes} contain the applicable pointers to any children and argument \textit{Nodes}. Hence this output can be used to form a tree, of the given \LaTeX file, in a form which Python can more-easily process further.
		
		One of the primary concepts behind this parser is that we leave all the direct logic required to capture the details of a \LaTeX command to the LeafNode class, this allows us to further extend the parser quite significantly without overcomplicating the functionality of our \textit{parse\_latex} method. This logic required to capture a LaTeX command will be discussed later on.
		
		Now, onto how \textit{parse\_latex} performs its task:
		it begins by iterating through each of the Nodes which have been appended to \textit{NodeBank}, checking with that Node if there exists any occurring matches inside of the text which was passed into \textit{parse\_latex}. These collected matches are then sorted into ascending order of appearance.
		
		Next, we begin to iterate through our sorted list of collected matches.
		During each iteration, we work out the current match's parent \textit{Node}. This is done using the stack data-structure and a while-loop.
		
		Each time we escape out the content of a particular \textit{Node}, we pop it out from the stack; creating a new \textit{TextNode} whose plain-text content is all the text bounded between our current match and our previous match.
		
		Hence the current match's parent \textit{Node} is the \textit{Node} obtained by performing a peek on the stack (item 'on-top' of the stack). In the case where the stack is empty, we know that the current match must be an immediate child of the inputted \LaTeX.
		
		Next, we must check whether the parent \textit{Node} is allowed children - in the case where the stack is empty, we assume this to always be true. We must also then check that the command for the current match does not occur prior to the start of the parent match.

		We are now able to create an instance of the sub-class of \textit{LeafNode} for the current match and, if applicable, we iterate through the current match's arguments, confirming that they're valid.
		
		We now build a tree for each valid argument's contents by making a recursive call to \textit{parse\_latex}; parsing the current valid argument's text content. If the returned array is not empty, we instantiate an \textit{ArgumentNode} and add all the Nodes in the returned array as children to our newly instanced \textit{ArgumentNode}.
		
		All of these instanced \textit{ArgumentNodes} are then added, as children, to our \textit{Node} for the current match.
		
		Similar to earlier, regarding the stack used to identify the parent \textit{Node} for our current match, we create a new \textit{TextNode} and set its content to be all the plain-text bounded between our current \textit{Node} and either the last match, or the most-recently popped \textit{Node} from the stack.
		
		Assuming the newly created \textit{TextNode} contains valid content, we shall append it as a child to the \textit{Node} obtained through performing a 'peek' onto the stack. In the case where the stack is empty, we simply append it to the array containing the immediate children of the passed \LaTeX.
		
		This above reasoning is then applied onto the \textit{Node} instance of the current match. We can now push our \textit{Node} instance onto the stack.
		
		The entire above process is repeated until we have iterated over each of our captured match. Once completed, we iteratively pop the remaining \textit{Nodes} from the stack and create \textit{TextNodes} for the remaining plain-text which had been missed.
		
		Finally, when the stack has been emptied, we create our last \textit{TextNode} for all, if any, plain-text that it remains within our passed \LaTeX.
		
		Below we have an implemented, and documented, Python code-listing for the process described above:
		\custpythonlstinputlisting{resources/parselatex.py}
		
	\subsubsection*{\LaTeX Parser's \textit{CommandNode}}
		As per our design, the \textit{CommandNode} is used to capture \LaTeX commands such as '\verb\textbf{Example}\' and '\verb\ref{ExampleReference}\'.
		
		This is a much simpler task than that described previously (\fullref{sssec:texbookparser}).
		
		The logic is as follows:
		We shall use regular expressions to identify matches for the start-string of our \LaTeX commands e.g. \\textbf{ and \\ref{. Next, we apply a simple bracket-counter which allows us to correctly identify the matching closing bracket.
				
		The process for this bracket-counter is described in the following steps:
		\begin{enumerate}
			\item Initiate a counter at 1 - as our regular expression's start-string contains '\{'.
			\item Iterate through each character occurring after the start-string's match end-position.
			\item If the current character is '\{' increment the counter by 1. Otherwise, if the current character is '\}' decrement the counter by 1. Making sure the previous character is not an escape character.
			\item If the counter is 0, we have identified the closing-braces, and the \LaTeX command's content is that bounded between the start-braces and closing-braces. We can now discontinue our iteration.
			\item If we have finished our iteration without encountering a matching closing-braces, there is an issue regarding the input text.
		\end{enumerate}
		
		The following details about the enclosed content of each match are stored in one of Python's namedtuple types, bearing the structure as follows:
			namedtuple("PreambleCapture", """
				inner\_start\_index, inner\_end\_index,
				outer\_start\_index, outer\_end\_index,
				arguments
			""")
		
		An array of these namedtuples are then returned to the parser.
		
		The implementation of this process is shown below:
		\custpythonlstinputlisting{resources/commandnode.py}
		
	\subsubsection*{\LaTeX Parser's \textit{EnvironmentNode}}
		\textit{EnvironmentNode} solves a similar problem to \textit{CommandNode}, however it is instead used to capture \LaTeX commands such as:
		\begin{verbatim}
			\begin{itemize} 
				\item First item 
				\item Second item
			\end{itemize},
			
			\begin{center}
				Hello World!
			\end{center},
		\end{verbatim}
		
		Clearly, the logic required to capture the contents of an \textit{EnvironmentNode} is different than that for a \textit{CommandNode}. The text bound between the \\begin and \\end commands is the content that the current command contains; thus the content we need to capture.
		
		The \textit{EnvironmentNode} utilises the same storage and returning procedure as that defined for the \textit{CommandNode}.
		
		The implementation of this process is shown below:
		\custpythonlstinputlisting{resources/environmentnode.py}
		
	\subsubsection*{\LaTeX Parser's \textit{LevelNode}}
		Similarly, to both \textit{CommandNode} and \textit{EnvironmentNode}, \textit{LevelNode} is used to capture \LaTeX commands such as:
		\begin{verbatim}
			\chapter{Example Chapter}
				\section{Example Section}
					\subsection{Example Subsection}
						\subsubsection{Example Subsubsection}
		\end{verbatim}
		
		As again, the logic required to capture the contents of an \textit{LevelNode} is different than that for a \textit{CommandNode} and am \textit{EnvironmentNode}. The text that is needed to be captured lies within the bounds of the current \textit{LevelNode} command and a later define \textit{LevelNode} of an equal to, or greater than hierarchical meaning.
		
		The \textit{LevelNode} utilises the same storage and returning procedure as that defined for the \textit{CommandNode}.
		
		The implementation of this process is shown below:
		\custpythonlstinputlisting{resources/levelnode.py}
		
	\subsection*{Evaluation}
	From what we have developed so far, we believe that we are on track to fully re-create the system as it was given. However, we acknowledge that this is a bold move in taking re-coding all of the system. We decided to take this challenge on board to deliver a better system that was originally provided but this comes at a cost with development time being cut on additional functionality. To date, we have successfully got the parser working, though more work will be needed to ensure that much of the functionality can be re-created. Overall, I feel that the requirements that we have proposed can be completed, but the extent that newer functionality is completed will depend on the time frame that is left after re-building CAMEL.\\       